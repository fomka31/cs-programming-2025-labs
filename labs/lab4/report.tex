\documentclass[]{vvsu}

\vvsuyear{2025}

%%%%%%%%%%%%%%%%%%%

\usepackage{graphicx} % для изображений
\usepackage{tabularray} % для таблиц
\usepackage{siunitx} % для обозначений (процент, градус)
\usepackage{listings} % для листингов кода

% Список путей, где будут искаться изображения и файлы
\graphicspath{{images/}}

% Автор документа
\author{Ф.Р.Кучерчук}

% Настройка стилей для листингов кода
\input{listing_styles.tex}

%%%%%%%%%%%%%%%%%%%

\begin{document}

% Шапка
\vvsuhead{\linespread{1}\selectfont{}МИНОБРНАУКИ РОССИИ\\
\vspace{10pt}Федеральное государственное бюджетное образовательное учреждение\\
высшего образования\\
\fontsize{13}{13}\selectfont{}<<ВЛАДИВОСТОКСКИЙ ГОСУДАРСТВЕННЫЙ УНИВЕРСИТЕТ>>\\
(ФГБОУ ВО <<ВВГУ>>)\\
\vspace{10pt}\fontsize{12}{12}\selectfont{}ИНСТИТУТ ИНФОРМАЦИОННЫХ ТЕХНОЛОГИЙ И АНАЛИЗА ДАННЫХ\\
КАФЕДРА ИНФОРМАЦИОННЫХ ТЕХНОЛОГИЙ И СИСТЕМ}

% Название отчета
\title{Отчет\\по лабораторной работе №4}
\subtitle{по дисциплине\\<<Информатика и программирование>>}

% Участники работы
\member{Студент\\ гр. БИН-25-2}{К.Ф. Кучерчук}
\member{Ассистент\\ преподавателя}{М.В. Водяницкий}

% Вывод титульника
\maketitle

% Задание
\begin{addition}{Задание}
  Выполнить задания и оформить отчет по стандартам ВВГУ.

  \textit{\textbf{Задание 1.}}  
  Написать программу, которая определяет, как будет вести себя кондиционер. Если температура в помещении 20 градусов и выше, то кондиционер выключается, если меньше - включается. Температура должна вводится пользователем с консоли.

  Пример:\\
    Введите температуру: 18\\
    Кондиционер включен

  \textit{\textbf{Задание 2.}}  
  Год делится на четыре сезона: зима, весна, лето и осень. Написать программу, которая запрашивает у пользователя номер месяца и выводит к какому сезону этот месяц относится.

  Пример:\\
    Введите номер месяца: 4\\
    Это весна 

  \textit{\textbf{Задание 3.}}  
  Считается, что один год, прожитый собакой, эквивалентен семи человеческим годам. При этом зачастую не учитывается, что собаки становятся абсолютно взрослыми уже к двум годам. Таким образом, многие предпочитают каждый из первых двух лет жизни собаки приравнивать к 10.5 годам человеческой жизни, а все последующие к 4.
  
  Написать программу, которая будет переводить собачий возраст в человеческий. Программа должна корректно обрабатывать входные данные и выводить соответствующие сообщения об ошибках:

  \begin{vvsu_itemize}
    \item Если вводится не число
    \item Если вводится число меньше 1
    \item Если вводится число большее 22
  \end{vvsu_itemize}

  Пример:\\
    Введите возраст собаки (в годах): 5\\
    Возраст собаки в человеческих годах: 33.0

  Пример:\\
    Введите возраст собаки (в годах): 0\\
    Ошибка: возраст должен быть не меньше 1

  \textit{\textbf{Задание 4.}}  
  Число делиться на 6 только в случае соблюдения двух условий:
  \begin{vvsu_itemize}
    \item Последняя цифра четная
    \item Сумма всех цифр делиться на 3
  \end{vvsu_itemize}
  Написать программу, которая выведет делиться ли введенное число на 6 или нет.

  \textit{\textbf{Задание 5.}}  
  Написать программу, которая будет проверять пароль на надежность. Пароль считается надежным, если его длина не менее 8 символов и если он содержит:

  \begin{vvsu_itemize}
    \item Заглавные буквы латиницы
    \item Строчные буквы латиницы
    \item Числа
    \item Специальные знаки
  \end{vvsu_itemize}

  В случае, если пароль не проходит по одному из условий, необходимо сообщить пользователю каким именно условиям он не удовлетворяет.

  Пример:\\
    Введите пароль: qwerty\\
    Пароль ненадежный: отсутствуют заглавные буквы, числа и специальные символы

  \textit{\textbf{Задание 6.}}  
  Написать программу, которая определяет, является ли введенный пользователем год високосным. Год считается високосным, если он делится на 4, но не делится на 100, либо если он делится на 400.

  Пример:\\
    Введите год: 2024\\
    2024 - високосный год

  \textit{\textbf{Задание 7.}}  
  Написать программу, которая запрашивает у пользователя три числа и выводит на экран наименьшее из них. При решении нельзя использовать встроенные функции min() и max().

  Пример:\\
    Введите три числа: 8 3 5\\
    Наименьшее число: 3

  \textit{\textbf{Задание 8.}}  
  В магазине проводится акция. Акция работает по следующим правилам:

  \begin{vvsu_itemize}
    \item Сумма < 1000 => скидка - 0\%
    \item Сумма < 5000 => скидка - 5\%
    \item Сумма < 10000 => скидка - 10\%
    \item Сумма > 10000 => скидка - 15\%
  \end{vvsu_itemize}

  Напишите программу, которая запрашивает сумму покупки и выводит размер скидки и итоговую сумму к оплате.
  
  Пример:\\
    Введите сумму покупки: 7500\\
    Ваша скидка: 10%\\
    К оплате : 6750.0
  
  \textit{\textbf{Задание 9.}}  
  Написать программу, которая определяет время суток по введенному часу (целое число от 0 до 23).

  \begin{vvsu_itemize}
    \item С 0 до 5 часов - ночь
    \item С 6 до 11 часов - утро
    \item С 12 до 17 часов - день
    \item С 18 до 23 часов - вечер
  \end{vvsu_itemize}
  
  Пример:\\
    Введите час (0–23): 20\\
    Сейчас вечер

  \textit{\textbf{Задание 10.}}  
  Написать программу, которая определяет, является ли введенное число простым. Число называется простым, если оно больше 1 и делится только на 1 и само себя. Программа должна корректно обрабатывать некорректный ввод и выводить соответствующие сообщения об ошибках.
  
  Пример:\\
    Введите число: 17\\
    17 - простое число
\end{addition}


% Содержание
\toc

% Глава - Выполнение работы
\section{Выполнение работы}

% Подглава - Задание 1
\subsection{Задание 1}

В переменную temp при помощи функции input() вносим вводимые пользователем данные, сразу переводя их в целочисленный тип данных функцией int(). После при помощи услоной конструкиции if else проверяем переменную temp на соответствие условиям: если temp больше или рано 20 (при помощи оператора сравнения больше равно >=), то мы при помощи функции print() выводим пользователю сообщение 'Кондиционер включается', в противном же случае при помощи той же функции выводится сообщение 'Кондиционер выключается'. На рисунке \ref{fig:code_task_1} представлен код программы.

\begin{vvsu_figure}{Листинг программы для задания 1}{fig:code_task_1}
  \begin{minipage}{.75\textwidth}
    \lstinputlisting[language=Python,basicstyle=\fontsize{10}{10}\linespread{1}\selectfont\ttfamily]{code/task1.py}
  \end{minipage}
\end{vvsu_figure}

% Подглава - Задание 2
\subsection{Задание 2}

Запрашиваем у пользователя на ввод строку (функция input()) и переводим её в число (функция int()). После при помощи условной конструкиции if else проверяем, к какому диапозону значений принадлежит переменная num, и выводим соответствующее сообщение. На рисунке \ref{fig:code_task_2} представлен код программы.

\begin{vvsu_figure}{Листинг программы для задания 2}{fig:code_task_2}
  \begin{minipage}{.75\textwidth}
    \lstinputlisting[language=Python,basicstyle=\fontsize{10}{10}\linespread{1}\selectfont\ttfamily]{code/task2.py}
  \end{minipage}
\end{vvsu_figure}

% Подглава - Задание 3
\subsection{Задание 3}

Создаём переменную dog-age и вносим её значение, вводимое пользователем с консоли при помощи функции input(). После для обработки ошибок используем конструкцию try expect. Внутри блока try пробуем перевести переменную dog-age к целочисленному типу данных. Если у нас получается, то код продолжает работать, а если же мы не можем привести её к этому типу данных, обработчик ошибок перемещает нас из блока try в блок expect, внутри которого мы при помощи функции print() выводим сообщение о соответствующей ошибке. Дальше внутри блока try мы создаём новую переменную current-age и присваиваем ей значение переменной dog-age, и создаём ещё одну переменную human-age, чтобы сохранять в неё возраст человека. После мы блоком if else проверяем входит ли значение переменной dog-age в область допустимых значений, и если нет то выводим соответствующее сообщение об ощибке. Если зпеременная удовлетворяет ОДС, то мы при помощи цикла while производим основные вычисления. Счётчиком будет являться переменная current-age, которую мы будем уменьшать в конце каждого цикла. Строками 13 и 14 мы показываем, что первые 2 года собаки считаются по 10.5 человеческх лет, остальные же по 4. И после заверения цикла while, а именно когда переменная current-age приравняется к нулю, мы при помощи функции print() выводим результат, записанный в переменную human-age. На рисунке \ref{fig:code_task_3} представлен код программы.

\begin{vvsu_figure}{Листинг программы для задания 3}{fig:code_task_3}
  \begin{minipage}{.75\textwidth}
    \lstinputlisting[language=Python,basicstyle=\fontsize{10}{10}\linespread{1}\selectfont\ttfamily]{code/task3.py}
  \end{minipage}
\end{vvsu_figure}

% Подглава - Задание 4
\subsection{Задание 4}

Создаё переменную num, в которую при помощи функции input() заносим переменную, и при помощи функции int() сразу приводим её к целочисленному типу данных. Дальше определяем функцию div-by-six, которая принимает 1 аргумент целочисленного типа данных, и возвращает булево значение. Дальше создаём переменую last-is-even, содержащую тернарный оперетор, который возвращает True если выполныется указанное условие и False если нет. Контекст этого условия следующий: мы должный проверить число на чётность, для этого достаточно чтобы последняя цифра числа делилась на два без остатка. Для этого мы приводим переменную num к строке функцией str(), затем при помощи индекса[-1] получаем последний символ и функцией int() приводим его обратно к строке и после этого сравниваем его остаток от делания на два с нулём. В следующий строке мы создаём ещё одну переменную summ-div-3, содержащую тернарный оперетор, который возвращает True если выполныется указанное условие и False если нет. Контекст условия: мы проверяем делится ли число на 3. Для этого необходимо чтобы сумма всех его чисел тоже далилась на три. Для получения суммы мы используем функцию sum(), которая возвращает число сумму, полученное сложение переданых в неё аргументов целочисленного типа данных. Таким образом мы помещаем в неё генератор int(x) for x in list(str(num)). Он работает следующим образом: мы берём x из списка, который мы получили путём приведения переменная num к строке функцией str() и последующим приведением её к списку функцией list(), после чего все x приводятся к целым числам и попадают в аргументы функции sum(). Потом сравниваем остаток от деления на три возвращённого ею значения с нулём. В конце функции возвращаем истину если переменные last-is-even и sum-div-3 имеют значение True и ложь в противном случае. В конце мы выводим сообщение пользователю при помощи f строки. На рисунке \ref{fig:code_task_4} представлен код решения. На рисунке \ref{fig:code_task_4} представлен код решения.

\begin{vvsu_figure}{Листинг программы для задания 4}{fig:code_task_4}
  \begin{minipage}{.75\textwidth}
    \lstinputlisting[language=Python,basicstyle=\fontsize{10}{10}\linespread{1}\selectfont\ttfamily]{code/task4.py}
  \end{minipage}
\end{vvsu_figure}


% Подглава - Задание 5
\subsection{Задание 5}

Запрашиваем ввод пароля от пользователя и записываем в переменную password. Создаем словарь с условиями проверки пароля: 
Пароль не менее 8 символов; наличие заглавных букв латиницы (any() возвращает True, если хотя бы один символ удовлетворяет условию, c.isupper() проверяет, является ли символ заглавной буквой); наличие строчных букв латиницы; наличие цифр (c.isdigit() проверяет, является ли символ цифрой); проверка наличия специальных символов (not c.isalnum() проверяет, НЕ является ли символ буквой или цифрой). Проверяем каждое условие с помощью цикла (items() возвращает пары (ключ, значение) из словаря).
 Если хотя бы одно условие не выполнено, общий результат становится False – пароль ненадежный. На рисунке \ref{fig:code_task_5} представлен код программы.

\begin{vvsu_figure}{Листинг программы для задания 5}{fig:code_task_5}
  \begin{minipage}{.75\textwidth}
    \lstinputlisting[language=Python,basicstyle=\fontsize{10}{10}\linespread{1}\selectfont\ttfamily]{code/task5.py}
  \end{minipage}
\end{vvsu_figure}

% Подглава - Задание 6
\subsection{Задание 6}

Год является високосным, если делится на 4 без остатка и не делится на 100 или делится на 400. Проверяем делимость. Создаём функцию visokos с одним аргументом god. После после при помощи операторов сравнения и логики проверяем переменную на удовлетворение условиям, и f строкой выводим результат.  На рисунке \ref{fig:code_task_6} представлен код программы.

\begin{vvsu_figure}{Листинг программы для задания 6}{fig:code_task_6}
  \begin{minipage}{.75\textwidth}
    \lstinputlisting[language=Python,basicstyle=\fontsize{10}{10}\linespread{1}\selectfont\ttfamily]{code/task6.py}
  \end{minipage}
\end{vvsu_figure}

% Подглава - Задание 7
\subsection{Задание 7}

Заводим список a, который наполняем числами, полученные путём перевода строки, разделённой методом split() и по частям переведённой в числа функцией int(), оборачивая всё это генератором. В следующей строке задаём минимальное значение по умолчанию, равное первому элементу списка. После при помощи цикла for проходимся по списку, начиная со второго элеманта при помощи среза [1:]. После проверяем тернарным оператором больше ли i нашего минимального значения, если меньше то переприсваиваем переменную minimal, в ином случае оставляем её неизменной. На рисунке \ref{fig:code_task_7} представлен код программы.

\begin{vvsu_figure}{Листинг программы для задания 7}{fig:code_task_7}
  \begin{minipage}{.75\textwidth}
    \lstinputlisting[language=Python,basicstyle=\fontsize{10}{10}\linespread{1}\selectfont\ttfamily]{code/task7.py}
  \end{minipage}
\end{vvsu_figure}

% Подглава - Задание 8
\subsection{Задание 8}

Для проверки ошибки открыаем блок try expect. В блоке try создаём переменную num и пробуем привести её к целочисленному типу данных, в случае ошибки на этом этапе нас выбрасывает в блок expect и выводится сообщение о соответствующей ошибке. Дальше внутри блока try заводим переменную discount для занесения в неё размера скидки, дальше проверяем принадлежность переменной num диапозонам значений при помощи условных операторов, при значении num меньше 1000 присваиваем переменной discount значение 0, при num больше или равное тысячи и меньше пяти тысяч присваиваем переменной скидки значение 5, при num больше или равно десяти тысячам делаем discount равным 10, при num больше десяти тысяч делаем скидку равной 15 процентам, и последний диапозон при num меньше или равном нулю, в таком случае мы вызываем ошибку вручную и снова переходим в блок expect. Дальше выводим на экран нашу скидку print() ом и f строкой. На следующей строке таким же образом расчитываем по формуле num*((100-discount)/100) финальную стоимость и выводим её в нужном формате. На рисунке \ref{fig:code_task_8} представлен код программы.

\begin{vvsu_figure}{Листинг программы для задания 8}{fig:code_task_8}
  \begin{minipage}{.75\textwidth}
    \lstinputlisting[language=Python,basicstyle=\fontsize{10}{10}\linespread{1}\selectfont\ttfamily]{code/task8.py}
  \end{minipage}
\end{vvsu_figure}

% Подглава - Задание 9
\subsection{Задание 9}

Создаём переменную time и заносим в ней значение, вводимое пользователем и переведённое в число, выведя при этом сообщение о вводе. Дальше через условные операторы проверяем переменную time на принадлежность диапозонам, создаваемых каждый своим генератором. Если time выходит за границы диапозона времени поднимаем ошибку, приводящее нас в блок expect и сообщением об ошибке ввода. Дальше в зависимости от принадлежности мы выводим сообщение:
при time принадлежащем int(i) for i in range(6) print('Сейчас ночь') 
при time принадлежащем int(i) for i in range(6,12) print('Сейчас утро') 
при time принадлежащем int(i) for i in range(12,18) print('Сейчас день') 
при time принадлежащем int(i) for i in range(18,24) print('Сейчас вечер'). 
На рисунке \ref{fig:code_task_9} представлен код программы.

\begin{vvsu_figure}{Листинг программы для задания 9}{fig:code_task_9}
  \begin{minipage}{.75\textwidth}
    \lstinputlisting[language=Python,basicstyle=\fontsize{10}{10}\linespread{1}\selectfont\ttfamily]{code/task9.py}
  \end{minipage}
\end{vvsu_figure}

% Подглава - Задание 10
\subsection{Задание 10}

Сначала определяется функция is-prime(number), которая проверяет, является ли число простым. В первой строке функции проверяется базовый случай: если число равно 0 или 1, оно не является простым, и функция возвращает False. Далее в цикле for перебираются все числа от 2 до number-1, и если находится делитель без остатка, функция немедленно возвращает False, указывая на составное число. Если по завершении цикла делителей не найдено, функция возвращает True, подтверждая простоту числа. Основная часть программы начинается с блока try, где запрашивается ввод пользователя, который преобразуется в целое число. Полученное число передается в функцию is-prime, и результат сохраняется в переменной res. Затем с помощью f-строки выводится форматированный результат, где тернарный оператор выбирает между вариантами "простое число" и "составное число" в зависимости от значения res. Если пользователь вводит нечисловые данные, срабатывает блок except ValueError, выводящий сообщение "Неверный ввод". На рисунке \ref{fig:code_task_10} представлен код программы.

\begin{vvsu_figure}{Листинг программы для задания 10}{fig:code_task_10}
  \begin{minipage}{.75\textwidth}
    \lstinputlisting[language=Python,basicstyle=\fontsize{10}{10}\linespread{1}\selectfont\ttfamily]{code/task10.py}
  \end{minipage}
\end{vvsu_figure}

Спасибо за понимание !

\end{document}