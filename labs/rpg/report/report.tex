\documentclass[]{vvsu}

\vvsuyear{2026}

%%%%%%%%%%%%%%%%%%%

\usepackage{graphicx}
\usepackage{tabularray}
\usepackage{siunitx}
\usepackage{listings}
\usepackage[utf8]{inputenc}
\usepackage[russian]{babel}
\usepackage{enumitem}
\usepackage{setspace}

\graphicspath{{images/}}

\author{Ф.Р. Кучерчук}

\input{listing_styles.tex}

% Настройка списков для лучшего отображения многоуровневых пунктов
\setlist[itemize]{left=0pt}
\setlist[enumerate]{left=0pt}

%%%%%%%%%%%%%%%%%%%

\begin{document}

% Шапка
\vvsuhead{\linespread{1}\selectfont{}МИНОБРНАУКИ РОССИИ\\
\vspace{10pt}Федеральное государственное бюджетное образовательное учреждение\\
высшего образования\\
\fontsize{13}{13}\selectfont{}<<ВЛАДИВОСТОКСКИЙ ГОСУДАРСТВЕННЫЙ УНИВЕРСИТЕТ>>\\
(ФГБОУ ВО <<ВВГУ>>)\\
\vspace{10pt}\fontsize{12}{12}\selectfont{}ИНСТИТУТ ИНФОРМАЦИОННЫХ ТЕХНОЛОГИЙ И АНАЛИЗА ДАННЫХ\\
КАФЕДРА ИНФОРМАЦИОННЫХ ТЕХНОЛОГИЙ И СИСТЕМ}

% Название отчета
\title{Отчет\\по разработке консольной rpg игры на python}
\subtitle{по дисциплине\\<<Информатика и программирование>>}

% Участники работы
\member{Студент\\ гр. БИН-25-2}{Ф.Р. Кучерчук}
\member{Ассистент\\ преподавателя}{М.В. Водяницкий}

% Вывод титульника
\maketitle

% Задание — полностью встроенный README
\begin{addition}{Задание}
  \onehalfspacing

  \textbf{Техническое задание — Текстовая RPG-игра}

  Вы работаете программистом в небольшой японской компании на заре игровой индустрии.  
  Компания разрабатывает свою первую экспериментальную игру — текстовую RPG, которая должна запускаться прямо в консоли и погружать игрока в атмосферу подземелий, опасностей и развития персонажа.

  Ваша задача — реализовать прототип игры, который демонстрирует основные игровые механики: характеристики персонажа, бои, прокачку, инвентарь и случайные события.

  \medskip

  \textbf{1. Общая идея программы}

  Программа представляет собой консольную текстовую RPG, в которой игрок:
  \begin{itemize}
    \item создает персонажа (выбор расы);
    \item получает случайные характеристики в рамках выбранной расы;
    \item исследует подземелье, состоящее из случайных комнат;
    \item сражается с врагами, находит предметы и улучшает персонажа;
    \item повышает уровень и распределяет очки характеристик;
    \item принимает решения, влияющие на дальнейший путь.
  \end{itemize}

  Игра работает в пошаговом режиме и управляется вводом команд с клавиатуры.

  \medskip

  \textbf{2. Создание персонажа}

  \textbf{2.1 Выбор расы}

  В начале игры пользователь выбирает расу персонажа (например):
  \begin{itemize}
    \item Человек
    \item Эльф
    \item Дворф
  \end{itemize}

  Каждая раса задает диапазоны генерации характеристик.

  \textbf{2.2 Характеристики персонажа}

  Характеристики генерируются случайным образом при создании персонажа, но в допустимых пределах для выбранной расы.

  Пример набора характеристик (можно расширять):
  \begin{itemize}
    \item \textbf{HP} — здоровье
    \item \textbf{Attack} — сила атаки
    \item \textbf{Defense} — защита
    \item \textbf{Agility} — ловкость (влияет на уклонение)
    \item \textbf{Height} — рост
    \item \textbf{Weight} — вес
  \end{itemize}

  Допускается, что некоторые характеристики влияют друг на друга (например, рост и вес влияют на уклонение или скорость).

  \medskip

  \textbf{3. Опыт и уровни}
  \begin{itemize}
    \item Персонаж получает опыт за победу над врагами.
    \item При накоплении нужного количества опыта повышается уровень.
    \item Каждый новый уровень дает очки прокачки.
  \end{itemize}

  \textbf{3.1 Прокачка характеристик}

  Игрок может распределять очки вручную между характеристиками.

  Пример:
  \begin{itemize}
    \item +1 к атаке
    \item +2 к HP
    \item +1 к ловкости
  \end{itemize}

  Распределение очков выполняется в комнатах отдыха.

  \medskip

  \textbf{4. Инвентарь и экипировка}

  \textbf{4.1 Инвентарь}

  Инвентарь хранит предметы:
  \begin{itemize}
    \item зелья (лечение и др.)
    \item монеты
    \item оружие
    \item прочие предметы
  \end{itemize}

  Игрок может:
  \begin{itemize}
    \item просматривать инвентарь
    \item использовать предметы
    \item выбрасывать любые предметы
  \end{itemize}

  \textbf{4.2 Экипировка}

  В инвентаре должны быть отдельные слоты:
  \begin{itemize}
    \item оружие
    \item броня
  \end{itemize}

  Экипированные предметы влияют на характеристики персонажа.

  \medskip

  \textbf{5. Подземелье и комнаты}

  \textbf{5.1 Структура подземелья}
  \begin{itemize}
    \item Игра начинается в подземелье.
    \item Подземелье состоит из комнат.
    \item После каждой комнаты игрок выбирает путь:
    \begin{itemize}
      \item налево
      \item направо
    \end{itemize}
  \end{itemize}

  Развилка есть после каждой комнаты.

  \textbf{5.2 Типы комнат}

  Комнаты генерируются случайно:
  \begin{itemize}
    \item Боевая комната — бой с врагом
    \item Комната отдыха — без событий
    \item Комната с сундуком — предметы или золото
  \end{itemize}

  Возможны комбинации:
  \begin{itemize}
    \item слева враг, справа сундук
    \item оба врага
    \item обе комнаты отдыха
  \end{itemize}

  \textbf{5.3 Видимость комнат}

  Перед выбором направления игрок:
  \begin{itemize}
    \item иногда знает, что находится дальше
    \item иногда не знает (темно, неизвестно)
  \end{itemize}

  Информация о видимости определяется случайно.

  \medskip

  \textbf{6. Враги и сложность}
  \begin{itemize}
    \item Враги генерируются случайно.
    \item У врагов есть характеристики (HP, атака, защита и т.д.).
    \item С каждым этажом подземелья сложность возрастает.
    \item Каждые \(N\) комнат или действий происходит переход на новый этаж.
  \end{itemize}

  \medskip

  \textbf{7. Боевая система}

  Бой происходит в пошаговом режиме:

  Пример действий игрока:
  \begin{itemize}
    \item атаковать
    \item использовать предмет
    \item попытаться уклониться
  \end{itemize}

  Учитываются:
  \begin{itemize}
    \item характеристики игрока
    \item экипировка
    \item случайные факторы (уклонение, критический удар)
  \end{itemize}

  \medskip

  \textbf{8. Предметы и добыча}
  \begin{itemize}
    \item Враги и сундуки могут давать:
    \begin{itemize}
      \item зелья
      \item оружие
      \item другие предметы
    \end{itemize}
    \item Полученные предметы добавляются в инвентарь.
    \item При нехватке места игрок решает, что выбросить.
  \end{itemize}

  \medskip

  \textbf{9. Хранение данных}

  Допускается (но не обязательно):
  \begin{itemize}
    \item сохранение состояния игры в файл
    \item использование формата \textbf{JSON} для хранения:
    \begin{itemize}
      \item характеристик персонажа
      \item инвентаря
      \item текущего этажа
    \end{itemize}
  \end{itemize}

  \medskip

  \textbf{10. Пример работы программы (фрагмент)}
  \begin{lstlisting}[basicstyle=\ttfamily\small]
Выберите расу:
1 - Человек
2 - Эльф
3 - Дворф

> 2

Ваш персонаж создан!
HP: 85
ATK: 12
DEF: 6
AFI: 14

Вы входите в подземелье...

Перед вами развилка.
(1) Слева: ???
(2) Справа: Комната отдыха

Куда пойти?
> 1
  \end{lstlisting}

  \medskip

  \textbf{11. Ограничения и требования}
  \begin{itemize}
    \item Программа \textbf{консольная}.
    \item Управление через текстовое меню и ввод команд.
    \item Язык программирования — не ограничен.
    \item Код должен быть читаемым и логически структурированным, можно делить на разные файлы.
  \end{itemize}

\end{addition}

% Содержание
\toc

% Введение
\section{Введение}

Разработка консольных текстовых игр представляет собой важный этап в обучении программированию, поскольку сочетает в себе работу с объектно-ориентированным проектированием, алгоритмами, пользовательским вводом и управлением состоянием. Такие проекты позволяют отработать навыки модульной архитектуры, обработки данных и взаимодействия с пользователем без необходимости использования графических библиотек.

Целью данной лабораторной работы является реализация прототипа текстовой RPG-игры, соответствующей техническому заданию, с акцентом на читаемость кода, логическую структуру и воспроизводимость игровых механик.

\subsection{Гейм-дизайн}

При проектировании игры были определены следующие ключевые принципы гейм-дизайна:

\begin{itemize}
  \item \textbf{Простота управления}: игра полностью управляется через текстовое меню с числовым вводом, что обеспечивает доступность и предсказуемость.
  \item \textbf{Прогрессия персонажа}: игрок ощущает рост силы через повышение уровня, увеличение характеристик и получение более мощного снаряжения.
  \item \textbf{Случайность и выбор}: каждый запуск игры уникален благодаря процедурной генерации комнат, врагов и предметов; при этом игрок сохраняет контроль над решением — куда идти, кого атаковать, что экипировать.
  \item \textbf{Баланс классов}: три класса (Воин, Лучник, Маг) имеют разные стартовые параметры и стратегии развития, что поощряет повторные прохождения.
\end{itemize}

Эти принципы легли в основу архитектурных и программных решений, принятых при реализации.

% Глава - Выполнение работы
\section{Выполнение работы}

\subsection{Архитектура программы}

Программа реализована в виде модульной структуры на языке Python. Основные компоненты:
\begin{itemize}
  \item \texttt{char\_classes.py} — класс \texttt{Hero}, управление характеристиками, уровнем, опытом;
  \item \texttt{enemy.py} — класс \texttt{Enemy} с генерацией параметров;
  \item \texttt{items\_classes.py} — классы \texttt{Item} и прочие;
  \item \texttt{dungeon.py} — логика генерации подземелья и комнат;
  \item \texttt{battle.py} — функции боевой системы;
  \item \texttt{main.py} — основной цикл игры и взаимодействие с пользователем.
\end{itemize}

\subsection{Создание персонажа и класса}

При запуске игры пользователь выбирает Класс. На основе выбора генерируются характеристики в заданных диапазонах. Например:
\begin{itemize}
  \item Воин: упор на живучесть;
  \item Лучник: дальний урон, выше ловкость;
  \item Маг: меньшая живучесть, но высокий урон.
\end{itemize}

\begin{vvsu_figure}{Фрагмент создания персонажа}{fig:char_creation}
  \begin{minipage}{.75\textwidth}
    \lstinputlisting[language=Python,basicstyle=\fontsize{10}{10}\linespread{1}\selectfont\ttfamily]{inserts/creation.py}
  \end{minipage}
\end{vvsu_figure}

\subsection{Система уровней и прокачки}

Опыт начисляется за победы. При достижении порога опыта персонаж повышает уровень и получает очки характеристик, которые распределяются автоматически в зависимости от класса игрока.
\begin{vvsu_figure}{Фрагмент повышения уровня}{fig:lvl_up}
  \begin{minipage}{.75\textwidth}
    \lstinputlisting[language=Python,basicstyle=\fontsize{10}{10}\linespread{1}\selectfont\ttfamily]{inserts/lvl.py}
  \end{minipage}
\end{vvsu_figure}

\begin{vvsu_figure}{Фрагмент повышения уровня воина}{fig:lvl_up}
  \begin{minipage}{.75\textwidth}
    \lstinputlisting[language=Python,basicstyle=\fontsize{10}{10}\linespread{1}\selectfont\ttfamily]{inserts/formelee.py}
  \end{minipage}
\end{vvsu_figure}

\subsection{Инвентарь и экипировка}

Инвентарь реализован как список объектов. Экипировка (оружие и броня) находится в отдельных слотах и модифицирует базовые характеристики (например, меч даёт +5 к Attack).

\begin{vvsu_figure}{Фрагмент класса оружия}{fig:weapon}
  \begin{minipage}{.75\textwidth}
    \lstinputlisting[language=Python,basicstyle=\fontsize{10}{10}\linespread{1}\selectfont\ttfamily]{inserts/weapon.py}
  \end{minipage}
\end{vvsu_figure}

\begin{vvsu_figure}{Фрагмент класса брони}{fig:armor}
  \begin{minipage}{.75\textwidth}
    \lstinputlisting[language=Python,basicstyle=\fontsize{10}{10}\linespread{1}\selectfont\ttfamily]{inserts/armor.py}
  \end{minipage}
\end{vvsu_figure}

\begin{vvsu_figure}{Фрагмент создания предмета}{fig:create}
  \begin{minipage}{.75\textwidth}
    \lstinputlisting[language=Python,basicstyle=\fontsize{10}{10}\linespread{1}\selectfont\ttfamily]{inserts/items.py}
  \end{minipage}
\end{vvsu_figure}

\subsection{Подземелье и генерация комнат}

Подземелье строится динамически. После каждой комнаты игрок выбирает направление. Содержимое соседних комнат может быть скрыто («???») или раскрыто («Комната отдыха»), в зависимости от случайного фактора.

\begin{vvsu_figure}{Фрагмент генерации комнат}{fig:rooms}
  \begin{minipage}{.75\textwidth}
    \lstinputlisting[language=Python,basicstyle=\fontsize{10}{10}\linespread{1}\selectfont\ttfamily]{inserts/rooms.py}
  \end{minipage}
\end{vvsu_figure}

\subsection{Подземелье и генерация комнат}

После прохождения определённого количества комнат на этаже игрок попадает в комнату босса, который представляет усиленную версию монстра.

\begin{vvsu_figure}{Фрагмент создания босса}{fig:boss1}
  \begin{minipage}{.75\textwidth}
    \lstinputlisting[language=Python,basicstyle=\fontsize{10}{10}\linespread{1}\selectfont\ttfamily]{inserts/bossCreate.py}
  \end{minipage}
\end{vvsu_figure}

\begin{vvsu_figure}{Фрагмент вызова босса}{fig:boss2}
  \begin{minipage}{.75\textwidth}
    \lstinputlisting[language=Python,basicstyle=\fontsize{10}{10}\linespread{1}\selectfont\ttfamily]{inserts/boss.py}
  \end{minipage}
\end{vvsu_figure}

\subsection{Сложность}

Сила врагов зависит от этажа на котором игрок с ним встречается.

\begin{vvsu_figure}{Фрагмент роста силы врагов}{fig:rostishka}
  \begin{minipage}{.75\textwidth}
    \lstinputlisting[language=Python,basicstyle=\fontsize{10}{10}\linespread{1}\selectfont\ttfamily]{inserts/scale.py}
  \end{minipage}
\end{vvsu_figure}

\subsection{Боевая система}

Бой реализован как цикл «ход игрока → ход врага». Игрок может атаковать, открыть инвентарь или попытаться уклониться. Вероятность уклонения зависит от уровня, но увы растёт только у лучника.

\begin{vvsu_figure}{Фрагмент боевой логики}{fig:combat}
  \begin{minipage}{.75\textwidth}
    \lstinputlisting[language=Python,basicstyle=\fontsize{10}{10}\linespread{1}\selectfont\ttfamily]{inserts/fight.py}
  \end{minipage}
\end{vvsu_figure}

\newpage

\subsection{Сохранение данных}

Состояние игры может быть сохранено в JSON-файл, включая:
\begin{itemize}
  \item характеристики персонажа,
  \item инвентарь и экипировку,
  \item текущий этаж и прогресс.
\end{itemize}

\begin{vvsu_figure}{Фрагмент сохранения}{fig:save}
  \begin{minipage}{.75\textwidth}
    \lstinputlisting[language=Python,basicstyle=\fontsize{10}{10}\linespread{1}\selectfont\ttfamily]{inserts/save.py}
  \end{minipage}
\end{vvsu_figure}

\begin{vvsu_figure}{Фрагмент загрузки}{fig:load}
  \begin{minipage}{.75\textwidth}
    \lstinputlisting[language=Python,basicstyle=\fontsize{10}{10}\linespread{1}\selectfont\ttfamily]{inserts/load.py}
  \end{minipage}
\end{vvsu_figure}

\newpage

% ТЕСТИРОВАНИЕ
\section{Тестирование}

Для проверки корректности работы всех компонентов игры было проведено ручное функциональное тестирование. Основные проверяемые сценарии:

\begin{itemize}
  \item Корректное создание персонажа с учётом выбранного класса и генерация характеристик в допустимых диапазонах.
  \item Правильное начисление опыта и повышение уровня после победы над врагом.
  \item Корректное применение эффектов экипировки: оружие увеличивает урон, броня — защиту.
  \item Генерация подземелья: все типы комнат (бой, отдых, сундук) появляются с ожидаемой частотой.
  \item Боевая система: урон зависит от атаки и защиты, уклонение работает только для Лучника и масштабируется с уровнем.
  \item Сохранение и загрузка: после перезапуска игры состояние персонажа восстанавливается без потерь.
\end{itemize}

Все протестированные сценарии завершились успешно. Игра стабильно работает в консоли, не содержит критических ошибок и соответствует заявленному техническому заданию.

\section{Заключение}

Разработан рабочий прототип консольной текстовой RPG, полностью соответствующий техническому заданию. Реализованы все ключевые механики: создание персонажа с расами, боевая система, прокачка, инвентарь, исследование подземелья и случайные события. Программа легко расширяема и соответствует требованиям к читаемости и модульности кода.

\end{document}