\documentclass[]{vvsu}

\vvsuyear{2026}

%%%%%%%%%%%%%%%%%%%

\usepackage{graphicx}
\usepackage{tabularray}
\usepackage{siunitx}
\usepackage{listings}
\usepackage[utf8]{inputenc}
\usepackage[russian]{babel}
\usepackage{enumitem}
\usepackage{setspace}

\graphicspath{{images/}}

\author{Ф.Р. Кучерчук}

\input{listing_styles.tex}

% Настройка списков: тире, выравнивание текста по левому краю основного текста
\setlist[itemize]{
  label=--,
  wide=\parindent,    % ← ключевая настройка: текст выровнен по левому краю основного текста
  topsep=4pt,
  itemsep=2pt,
  parsep=0pt
}
\setlist[enumerate]{
  wide=\parindent
}

%%%%%%%%%%%%%%%%%%%

\begin{document}

% Шапка
\vvsuhead{\linespread{1}\selectfont{}МИНОБРНАУКИ РОССИИ\\
\vspace{10pt}Федеральное государственное бюджетное образовательное учреждение\\
высшего образования\\
\fontsize{13}{13}\selectfont{}<<ВЛАДИВОСТОКСКИЙ ГОСУДАРСТВЕННЫЙ УНИВЕРСИТЕТ>>\\
(ФГБОУ ВО <<ВВГУ>>)\\
\vspace{10pt}\fontsize{12}{12}\selectfont{}ИНСТИТУТ ИНФОРМАЦИОННЫХ ТЕХНОЛОГИЙ И АНАЛИЗА ДАННЫХ\\
КАФЕДРА ИНФОРМАЦИОННЫХ ТЕХНОЛОГИЙ И СИСТЕМ}

% Название отчета
\title{Отчет\\по разработке консольной rpg игры на python}
\subtitle{по дисциплине\\<<Информатика и программирование>>}

% Участники работы
\member{Студент\\ гр. БИН-25-2}{Ф.Р. Кучерчук}
\member{Ассистент\\ преподавателя}{М.В. Водяницкий}

% Вывод титульника
\maketitle

% Задание — полностью встроенный README
\begin{addition}{Задание}
  \onehalfspacing

  Техническое задание — Текстовая RPG-игра

  Вы работаете программистом в небольшой японской компании на заре игровой индустрии.  
  Компания разрабатывает свою первую экспериментальную игру — текстовую RPG, которая должна запускаться прямо в консоли и погружать игрока в атмосферу подземелий, опасностей и развития персонажа.

  Ваша задача — реализовать прототип игры, который демонстрирует основные игровые механики: характеристики персонажа, бои, прокачку, инвентарь и случайные события.

  \medskip

  1. Общая идея программы

  Программа представляет собой консольную текстовую RPG, в которой игрок:
  \begin{itemize}
    \item создает персонажа (выбор расы);
    \item получает случайные характеристики в рамках выбранной расы;
    \item исследует подземелье, состоящее из случайных комнат;
    \item сражается с врагами, находит предметы и улучшает персонажа;
    \item повышает уровень и распределяет очки характеристик;
    \item принимает решения, влияющие на дальнейший путь.
  \end{itemize}

  Игра работает в пошаговом режиме и управляется вводом команд с клавиатуры.

  \medskip

  2. Создание персонажа

  2.1 Выбор расы

  В начале игры пользователь выбирает расу персонажа (например):
  \begin{itemize}
    \item Человек
    \item Эльф
    \item Дворф
  \end{itemize}

  Каждая раса задает диапазоны генерации характеристик.

  2.2 Характеристики персонажа

  Характеристики генерируются случайным образом при создании персонажа, но в допустимых пределах для выбранной расы.

  Пример набора характеристик (можно расширять):
  \begin{itemize}
    \item HP — здоровье
    \item Attack — сила атаки
    \item Defense — защита
    \item Agility — ловкость (влияет на уклонение)
    \item Height — рост
    \item Weight — вес
  \end{itemize}

  Допускается, что некоторые характеристики влияют друг на друга (например, рост и вес влияют на уклонение или скорость).

  \medskip

  3. Опыт и уровни
  \begin{itemize}
    \item Персонаж получает опыт за победу над врагами.
    \item При накоплении нужного количества опыта повышается уровень.
    \item Каждый новый уровень дает очки прокачки.
  \end{itemize}

  3.1 Прокачка характеристик

  Игрок может распределять очки вручную между характеристиками.

  Пример:
  \begin{itemize}
    \item +1 к атаке
    \item +2 к HP
    \item +1 к ловкости
  \end{itemize}

  Распределение очков выполняется в комнатах отдыха.

  \medskip

  4. Инвентарь и экипировка

  4.1 Инвентарь

  Инвентарь хранит предметы:
  \begin{itemize}
    \item зелья (лечение и др.)
    \item монеты
    \item оружие
    \item прочие предметы
  \end{itemize}

  Игрок может:
  \begin{itemize}
    \item просматривать инвентарь
    \item использовать предметы
    \item выбрасывать любые предметы
  \end{itemize}

  4.2 Экипировка

  В инвентаре должны быть отдельные слоты:
  \begin{itemize}
    \item оружие
    \item броня
  \end{itemize}

  Экипированные предметы влияют на характеристики персонажа.

  \medskip

  5. Подземелье и комнаты

  5.1 Структура подземелья
  \begin{itemize}
    \item Игра начинается в подземелье.
    \item Подземелье состоит из комнат.
    \item После каждой комнаты игрок выбирает путь:
    \begin{itemize}
      \item налево
      \item направо
    \end{itemize}
  \end{itemize}

  Развилка есть после каждой комнаты.

  5.2 Типы комнат

  Комнаты генерируются случайно:
  \begin{itemize}
    \item Боевая комната — бой с врагом
    \item Комната отдыха — без событий
    \item Комната с сундуком — предметы или золото
  \end{itemize}

  Возможны комбинации:
  \begin{itemize}
    \item слева враг, справа сундук
    \item оба врага
    \item обе комнаты отдыха
  \end{itemize}

  5.3 Видимость комнат

  Перед выбором направления игрок:
  \begin{itemize}
    \item иногда знает, что находится дальше
    \item иногда не знает (темно, неизвестно)
  \end{itemize}

  Информация о видимости определяется случайно.

  \medskip

  6. Враги и сложность
  \begin{itemize}
    \item Враги генерируются случайно.
    \item У врагов есть характеристики (HP, атака, защита и т.д.).
    \item С каждым этажом подземелья сложность возрастает.
    \item Каждые \(N\) комнат или действий происходит переход на новый этаж.
  \end{itemize}

  \medskip

  7. Боевая система

  Бой происходит в пошаговом режиме:

  Пример действий игрока:
  \begin{itemize}
    \item атаковать
    \item использовать предмет
    \item попытаться уклониться
  \end{itemize}

  Учитываются:
  \begin{itemize}
    \item характеристики игрока
    \item экипировка
    \item случайные факторы (уклонение, критический удар)
  \end{itemize}

  \medskip

  8. Предметы и добыча
  \begin{itemize}
    \item Враги и сундуки могут давать:
    \begin{itemize}
      \item зелья
      \item оружие
      \item другие предметы
    \end{itemize}
    \item Полученные предметы добавляются в инвентарь.
    \item При нехватке места игрок решает, что выбросить.
  \end{itemize}

  \medskip

  9. Хранение данных

  Допускается (но не обязательно):
  \begin{itemize}
    \item сохранение состояния игры в файл
    \item использование формата JSON для хранения:
    \begin{itemize}
      \item характеристик персонажа
      \item инвентаря
      \item текущего этажа
    \end{itemize}
  \end{itemize}

  \medskip

  10. Пример работы программы (фрагмент)
  \begin{vvsu_figure}{Фрагмент создания персонажа}{fig:char_creation}
    \begin{minipage}{.75\textwidth}
      \lstinputlisting[language=Python,basicstyle=\fontsize{10}{10}\linespread{1}\selectfont\ttfamily]{inserts/txt.py}
    \end{minipage}
  \end{vvsu_figure}

  \medskip

  11. Ограничения и требования
  \begin{itemize}
    \item Программа консольная.
    \item Управление через текстовое меню и ввод команд.
    \item Язык программирования — не ограничен.
    \item Код должен быть читаемым и логически структурированным, можно делить на разные файлы.
  \end{itemize}

\end{addition}

% Содержание
\toc

% Введение
\section{Введение}

Разработка консольных текстовых игр представляет собой важный этап в обучении программированию, поскольку сочетает в себе работу с объектно-ориентированным проектированием, алгоритмами, пользовательским вводом и управлением состоянием. Такие проекты позволяют отработать навыки модульной архитектуры, обработки данных и взаимодействия с пользователем без необходимости использования графических библиотек.

Целью данной лабораторной работы является реализация прототипа текстовой RPG-игры, соответствующей техническому заданию, с акцентом на читаемость кода, логическую структуру и воспроизводимость игровых механик.

\subsection{Гейм-дизайн}

При проектировании игры были определены следующие ключевые принципы гейм-дизайна:

\begin{itemize}
  \item Простота управления: игра полностью управляется через текстовое меню с числовым вводом, что обеспечивает доступность и предсказуемость.
  \item Прогрессия персонажа: игрок ощущает рост силы через повышение уровня, увеличение характеристик и получение более мощного снаряжения.
  \item Случайность и выбор: каждый запуск игры уникален благодаря процедурной генерации комнат, врагов и предметов; при этом игрок сохраняет контроль над решением — куда идти, кого атаковать, что экипировать.
  \item Баланс классов: три класса (Воин, Лучник, Маг) имеют разные стартовые параметры и стратегии развития, что поощряет повторные прохождения.
\end{itemize}

Эти принципы легли в основу архитектурных и программных решений, принятых при реализации.

% Глава - Выполнение работы
\section{Выполнение работы}

\subsection{Архитектура программы}

Программа реализована в виде модульной структуры на языке Python. Основные компоненты:
\begin{itemize}
  \item char\_classes.py — класс Hero, управление характеристиками, уровнем, опытом;
  \item enemy.py — класс Enemy с генерацией параметров;
  \item items\_classes.py — классы Item и прочие;
  \item dungeon.py — логика генерации подземелья и комнат;
  \item battle.py — функции боевой системы;
  \item main.py — основной цикл игры и взаимодействие с пользователем.
\end{itemize}

Такая структура позволяет легко расширять функционал, заменять отдельные модули и поддерживать читаемость кода. Каждый файл отвечает за одну зону ответственности, что соответствует принципам SOLID.

\subsection{Создание персонажа и класса}

При запуске игры пользователь выбирает Класс. На основе выбора генерируются характеристики в заданных диапазонах. Например:
\begin{itemize}
  \item Воин: упор на живучесть;
  \item Лучник: дальний урон, выше ловкость;
  \item Маг: меньшая живучесть, но высокий урон.
\end{itemize}

\begin{vvsu_figure}{Фрагмент создания персонажа}{fig:char_creation}
  \begin{minipage}{.75\textwidth}
    \lstinputlisting[language=Python,basicstyle=\fontsize{10}{10}\linespread{1}\selectfont\ttfamily]{inserts/creation.py}
  \end{minipage}
\end{vvsu_figure}

Приведённый фрагмент демонстрирует инициализацию объекта героя с учётом выбранного класса. Случайная генерация характеристик происходит в рамках, заданных для каждого класса, что обеспечивает баланс и разнообразие игровых проходов.

\subsection{Система уровней и прокачки}

Опыт начисляется за победы. При достижении порога опыта персонаж повышает уровень и получает очки характеристик, которые распределяются автоматически в зависимости от класса игрока.

\begin{vvsu_figure}{Фрагмент повышения уровня}{fig:lvl_up}
  \begin{minipage}{.75\textwidth}
    \lstinputlisting[language=Python,basicstyle=\fontsize{10}{10}\linespread{1}\selectfont\ttfamily]{inserts/lvl.py}
  \end{minipage}
\end{vvsu_figure}

Данный код реализует базовую логику повышения уровня: проверка накопленного опыта, увеличение уровня и вызов метода распределения очков. Это позволяет игроку ощущать прогресс по мере прохождения игры.

\begin{vvsu_figure}{Фрагмент повышения уровня воина}{fig:lvl_up}
  \begin{minipage}{.75\textwidth}
    \lstinputlisting[language=Python,basicstyle=\fontsize{10}{10}\linespread{1}\selectfont\ttfamily]{inserts/formelee.py}
  \end{minipage}
\end{vvsu_figure}

В этом примере показано, как именно распределяются очки для класса «Воин»: приоритет отдаётся здоровью и атаке, что соответствует роли танка в бою. Аналогичные функции существуют для других классов.

\subsection{Инвентарь и экипировка}

Инвентарь реализован как список объектов. Экипировка (оружие и броня) находится в отдельных слотах и модифицирует базовые характеристики (например, меч даёт +5 к Attack).

\begin{vvsu_figure}{Фрагмент класса оружия}{fig:weapon}
  \begin{minipage}{.75\textwidth}
    \lstinputlisting[language=Python,basicstyle=\fontsize{10}{10}\linespread{1}\selectfont\ttfamily]{inserts/weapon.py}
  \end{minipage}
\end{vvsu_figure}

Класс \texttt{Weapon} наследуется от базового \texttt{Item} и содержит дополнительное поле \texttt{attack\_bonus}, которое при экипировке суммируется с базовой атакой героя. Это упрощает расчёт урона в бою.

\begin{vvsu_figure}{Фрагмент класса брони}{fig:armor}
  \begin{minipage}{.75\textwidth}
    \lstinputlisting[language=Python,basicstyle=\fontsize{10}{10}\linespread{1}\selectfont\ttfamily]{inserts/armor.py}
  \end{minipage}
\end{vvsu_figure}

Аналогично, класс \texttt{Armor} добавляет бонус к защите. При смене экипировки старые бонусы корректно удаляются, а новые — применяются, что исключает ошибки в расчётах.

\begin{vvsu_figure}{Фрагмент создания предмета}{fig:create}
  \begin{minipage}{.75\textwidth}
    \lstinputlisting[language=Python,basicstyle=\fontsize{10}{10}\linespread{1}\selectfont\ttfamily]{inserts/items.py}
  \end{minipage}
\end{vvsu_figure}

Базовый класс \texttt{Item} обеспечивает единый интерфейс для всех предметов, включая зелья и ключи. Это упрощает работу с инвентарём и позволяет легко добавлять новые типы предметов.

\subsection{Подземелье и генерация комнат}

Подземелье строится динамически. После каждой комнаты игрок выбирает направление. Содержимое соседних комнат может быть скрыто («???») или раскрыто («Комната отдыха»), в зависимости от случайного фактора.

\begin{vvsu_figure}{Фрагмент генерации комнат}{fig:rooms}
  \begin{minipage}{.75\textwidth}
    \lstinputlisting[language=Python,basicstyle=\fontsize{10}{10}\linespread{1}\selectfont\ttfamily]{inserts/rooms.py}
  \end{minipage}
\end{vvsu_figure}

Функция генерации использует веса вероятностей для разных типов комнат, что позволяет контролировать частоту боёв, отдыха и сундуков. Также реализована логика видимости, создающая эффект неопределённости.

\subsection{Боссы и прогрессия по этажам}

После прохождения определённого количества комнат на этаже игрок попадает в комнату босса, который представляет усиленную версию монстра.

\begin{vvsu_figure}{Фрагмент создания босса}{fig:boss1}
  \begin{minipage}{.75\textwidth}
    \lstinputlisting[language=Python,basicstyle=\fontsize{10}{10}\linespread{1}\selectfont\ttfamily]{inserts/bossCreate.py}
  \end{minipage}
\end{vvsu_figure}

Босс создаётся как экземпляр класса \texttt{Enemy}, но с увеличенными характеристиками и уникальным именем. Это добавляет разнообразие и повышает сложность на каждом новом этаже.

\begin{vvsu_figure}{Фрагмент вызова босса}{fig:boss2}
  \begin{minipage}{.75\textwidth}
    \lstinputlisting[language=Python,basicstyle=\fontsize{10}{10}\linespread{1}\selectfont\ttfamily]{inserts/boss.py}
  \end{minipage}
\end{vvsu_figure}

Логика вызова босса активируется после счёта пройденных комнат. Игрок получает предупреждение, что впереди сильный противник, что даёт возможность подготовиться (использовать зелья, экипировать лучшее снаряжение).

\subsection{Сложность}

Сила врагов зависит от этажа на котором игрок с ним встречается.

\begin{vvsu_figure}{Фрагмент роста силы врагов}{fig:rostishka}
  \begin{minipage}{.75\textwidth}
    \lstinputlisting[language=Python,basicstyle=\fontsize{10}{10}\linespread{1}\selectfont\ttfamily]{inserts/scale.py}
  \end{minipage}
\end{vvsu_figure}

Коэффициент сложности линейно возрастает с каждым этажом. Это гарантирует, что игра остаётся вызовом даже на поздних стадиях, и побуждает игрока активно прокачивать персонажа.

\subsection{Боевая система}

Бой реализован как цикл «ход игрока → ход врага». Игрок может атаковать, открыть инвентарь или попытаться уклониться. Вероятность уклонения зависит от уровня, но увы растёт только у лучника.

\begin{vvsu_figure}{Фрагмент боевой логики}{fig:combat}
  \begin{minipage}{.75\textwidth}
    \lstinputlisting[language=Python,basicstyle=\fontsize{10}{10}\linespread{1}\selectfont\ttfamily]{inserts/fight.py}
  \end{minipage}
\end{vvsu_figure}

Боевая система учитывает текущее состояние героя: здоровье, экипировку, наличие зелий. Также реализованы базовые механики уклонения и критического урона, зависящие от ловкости и удачи.

\subsection{Сохранение данных}

Состояние игры может быть сохранено в JSON-файл, включая:
\begin{itemize}
  \item характеристики персонажа,
  \item инвентарь и экипировку,
  \item текущий этаж и прогресс.
\end{itemize}

Процесс сериализации преобразует объекты Python в словари, которые затем записываются в JSON. Это позволяет сохранять сложные структуры данных в человекочитаемом формате. При загрузке данные из JSON восстанавливаются в объекты соответствующих классов, что обеспечивает полное восстановление состояния игры без потерь.

Фрагменты кода функций сохранения и загрузки приведены в приложении~А.

% ТЕСТИРОВАНИЕ
\section{Тестирование}

Для проверки корректности работы всех компонентов игры было проведено ручное функциональное тестирование. Основные проверяемые сценарии:

\begin{itemize}
  \item Корректное создание персонажа с учётом выбранного класса и генерация характеристик в допустимых диапазонах.
  \item Правильное начисление опыта и повышение уровня после победы над врагом.
  \item Корректное применение эффектов экипировки: оружие увеличивает урон, броня — защиту.
  \item Генерация подземелья: все типы комнат (бой, отдых, сундук) появляются с ожидаемой частотой.
  \item Боевая система: урон зависит от атаки и защиты, уклонение работает только для Лучника и масштабируется с уровнем.
  \item Сохранение и загрузка: после перезапуска игры состояние персонажа восстанавливается без потерь.
\end{itemize}

Все протестированные сценарии завершились успешно. Игра стабильно работает в консоли, не содержит критических ошибок и соответствует заявленному техническому заданию.

\section{Заключение}

Разработан рабочий прототип консольной текстовой RPG, полностью соответствующий техническому заданию. Реализованы все ключевые механики: создание персонажа с расами, боевая система, прокачка, инвентарь, исследование подземелья и случайные события. Программа легко расширяема и соответствует требованиям к читаемости и модульности кода.

% ПРИЛОЖЕНИЯ
\begin{application}{Приложение А}{Приложения}

\onehalfspacing

Сохранение состояния игры

\begin{vvsu_figure}{Функция сохранения}{fig:save_app}
  \begin{minipage}{.75\textwidth}
    \lstinputlisting[language=Python,basicstyle=\fontsize{10}{10}\linespread{1}\selectfont\ttfamily]{inserts/save.py}
  \end{minipage}
\end{vvsu_figure}

\vspace{12pt}

\newpage

Загрузка состояния игры

\begin{vvsu_figure}{Функция загрузки}{fig:load_app}
  \begin{minipage}{.75\textwidth}
    \lstinputlisting[language=Python,basicstyle=\fontsize{10}{10}\linespread{1}\selectfont\ttfamily]{inserts/load.py}
  \end{minipage}
\end{vvsu_figure}

\end{application}

\end{document}