\documentclass[]{vvsu}

\vvsuyear{2026}

%%%%%%%%%%%%%%%%%%%

\usepackage{graphicx}
\usepackage{tabularray}
\usepackage{siunitx}
\usepackage{listings}
\usepackage[russian]{babel}

\graphicspath{{images/}}

\author{Кучерчук Ф.Р.}

\input{listing_styles.tex}

%%%%%%%%%%%%%%%%%%%

\begin{document}

% Шапка
\vvsuhead{\linespread{1}\selectfont{}МИНОБРНАУКИ РОССИИ\\
\vspace{10pt}Федеральное государственное бюджетное образовательное учреждение\\
высшего образования\\
\fontsize{13}{13}\selectfont{}<<ВЛАДИВОСТОКСКИЙ ГОСУДАРСТВЕННЫЙ УНИВЕРСИТЕТ>>\\
(ФГБОУ ВО <<ВВГУ>>)\\
\vspace{10pt}\fontsize{12}{12}\selectfont{}ИНСТИТУТ ИНФОРМАЦИОННЫХ ТЕХНОЛОГИЙ И АНАЛИЗА ДАННЫХ\\
КАФЕДРА ИНФОРМАЦИОННЫХ ТЕХНОЛОГИЙ И СИСТЕМ}

% Название отчета
\title{Отчёт\\по лабораторной работе}
\subtitle{по дисциплине\\<<Информатика и программирование>>}

% Участники работы
\member{Студент\\ гр. БИН-25-2}{Кучерчук Ф.Р.}
\member{Ассистент\\ преподавателя}{М.В. Водяницкий}

% Вывод титульника
\maketitle

% Задание
\begin{addition}{Задание}
# Техническое задание - Текстовая RPG-игра

Вы работаете программистом в небольшой японской компании на заре игровой индустрии.
Компания разрабатывает свою первую экспериментальную игру - текстовую RPG, которая должна запускаться прямо в консоли и погружать игрока в атмосферу подземелий, опасностей и развития персонажа

Ваша задача - реализовать прототип игры, который демонстрирует основные игровые механики: характеристики персонажа, бои, прокачку, инвентарь и случайные события

---

## 1. Общая идея программы

Программа представляет собой консольную текстовую RPG, в которой игрок:

* создает персонажа (выбор расы)
* получает случайные характеристики в рамках выбранной расы
* исследует подземелье, состоящее из случайных комнат
* сражается с врагами, находит предметы и улучшает персонажа
* повышает уровень и распределяет очки характеристик
* принимает решения, влияющие на дальнейший путь

Игра работает в пошаговом режиме и управляется вводом команд с клавиатуры

## 2. Создание персонажа

### 2.1 Выбор расы

В начале игры пользователь выбирает расу персонажа (например):

* Человек
* Эльф
* Дворф

Каждая раса задает диапазоны генерации характеристик

### 2.2 Характеристики персонажа

Характеристики генерируются случайным образом при создании персонажа, но в допустимых пределах для выбранной расы

Пример набора характеристик (можно расширять):

* **HP** - здоровье
* **Attack** - сила атаки
* **Defense** - защита
* **Agility** - ловкость (влияет на уклонение)
* **Height** - рост
* **Weight** - вес

Допускается, что некоторые характеристики влияют друг на друга
(например, рост и вес влияют на уклонение или скорость)

## 3. Опыт и уровни

* Персонаж получает опыт за победу над врагами
* При накоплении нужного количества опыта повышается уровень
* Каждый новый уровень дает очки прокачки

### 3.1 Прокачка характеристик

Игрок может распределять очки вручную между характеристиками

Пример:

* +1 к атаке
* +2 к HP
* +1 к ловкости

Распределение очков выполняется в комнатах отдыха

## 4. Инвентарь и экипировка

### 4.1 Инвентарь

Инвентарь хранит предметы:

* зелья (лечение и др.)
* монеты
* оружие
* прочие предметы

Игрок может:

* просматривать инвентарь
* использовать предметы
* выбрасывать любые предметы

### 4.2 Экипировка

В инвентаре должны быть отдельные слоты:

* оружие
* броня

Экипированные предметы влияют на характеристики персонажа

## 5. Подземелье и комнаты

### 5.1 Структура подземелья

* Игра начинается в подземелье
* Подземелье состоит из комнат
* После каждой комнаты игрок выбирает путь:

  * налево
  * направо

Развилка есть после каждой комнаты

### 5.2 Типы комнат

Комнаты генерируются случайно:

* Боевая комната - бой с врагом
* Комната отдыха - без событий
* Комната с сундуком - предметы или золото

Возможны комбинации:

* слева враг, справа сундук
* оба врага
* обе комнаты отдыха

### 5.3 Видимость комнат

Перед выбором направления игрок:

* иногда знает, что находится дальше
* иногда не знает (темно, неизвестно)

Информация о видимости определяется случайно.

## 6. Враги и сложность

* Враги генерируются случайно
* У врагов есть характеристики (HP, атака, защита и т.д.)
* С каждым этажом подземелья сложность возрастает
* Каждые `N` комнат или действий происходит переход на новый этаж

## 7. Боевая система

Бой происходит в пошаговом режиме:

Пример действий игрока:

* атаковать
* использовать предмет
* попытаться уклониться

Учитываются:

* характеристики игрока
* экипировка
* случайные факторы (уклонение, критический удар)

## 8. Предметы и добыча

* Враги и сундуки могут давать:

  * зелья
  * оружие
  * другие предметы
* Полученные предметы добавляются в инвентарь
* При нехватке места игрок решает, что выбросить

## 9. Хранение данных

Допускается (но не обязательно):

* сохранение состояния игры в файл
* использование формата **JSON** для хранения:

  * характеристик персонажа
  * инвентаря
  * текущего этажа

\end{addition}

% Содержание
\toc

% Введение
\section{Введение}

Разработка консольных игр остаётся важной частью обучения программированию, так как позволяет отработать ключевые концепции: объектно-ориентированное проектирование, управление состоянием, работу с файлами и взаимодействие с пользователем. Целью данной лабораторной работы является реализация прототипа текстовой RPG-игры на языке Python, демонстрирующей основные игровые механики: систему характеристик, боевую систему, инвентарь, прокачку персонажа и сохранение прогресса. Работа выполнена в соответствии с техническим заданием, предоставленным преподавателем.

% Архитектура
\section{Архитектура программного решения}

Проект реализован с использованием модульной архитектуры, обеспечивающей чёткое разделение ответственности между компонентами. Структура проекта представлена на рисунке~\ref{fig:arch}.

\begin{vvsu_figure}{Структура проекта}{fig:arch}
\begin{minipage}{0.9\linewidth}
\begin{verbatim}
rpg/
├── characters/      # Сущности: герой, враг
├── items/           # Предметная система
├── game/            # Игровой поток и подземелье
├── logic/           # Боевая логика
├── persistence/     # Сохранение/загрузка
├── utils/           # Вспомогательные функции
├── config/          # Константы
└── main.py          # Точка входа
\end{verbatim}
\end{minipage}
\end{vvsu_figure}

Основу архитектуры составляет иерархия классов:
\begin{itemize}
    \item \texttt{Entity} — абстрактная базовая сущность с характеристиками (HP, урон, защита).
    \item \texttt{Hero} — наследник \texttt{Entity}, расширенный инвентарём, опытом и методами управления экипировкой.
    \item \texttt{Melee}, \texttt{Ranger}, \texttt{Mage} — конкретные классы героев с уникальными стартовыми характеристиками и формулами роста.
    \item \texttt{Enemy} — простая сущность противника без инвентаря.
\end{itemize}

Взаимодействие между модулями осуществляется через чётко определённые интерфейсы. Например, модуль \texttt{battle.py} оперирует только объектами типа \texttt{Entity}, что позволяет проводить бои между любыми сущностями. Модуль \texttt{game\_flow.py} координирует работу всех компонентов, обеспечивая пользовательский ввод и вывод.

% Гейм-дизайн
\section{Гейм-дизайн}

Игра реализует классическую структуру текстовой RPG с акцентом на развитие персонажа и исследование процедурно генерируемого подземелья.

\subsection{Классы персонажей}

Реализованы три класса:
\begin{vvsu_itemize}
    \item \textbf{Воин (Melee)} — высокое здоровье и физический урон, низкая мана.
    \item \textbf{Лучник (Ranger)} — среднее здоровье, высокий пронзающий урон и шанс уклонения.
    \item \textbf{Маг (Mage)} — низкое здоровье, высокий магический урон и большое количество маны.
\end{vvsu_itemize}

Каждый класс имеет уникальные формулы расчёта характеристик и роста при повышении уровня.

\subsection{Подземелье и прогрессия}

Подземелье состоит из этажей. Каждый этаж содержит 5 комнат, после которых следует босс-битва. Комнаты генерируются случайно и могут быть трёх типов:
\begin{vvsu_itemize}
    \item \textbf{Боевая} — сражение с обычным врагом.
    \item \textbf{Отдых} — восстановление части здоровья.
    \item \textbf{Сундук} — получение случайного предмета из списка добычи.
\end{vvsu_itemize}

Перед входом в комнату игрок с вероятностью 60\% видит её тип, что добавляет элемент стратегического выбора.

\subsection{Боевая система}

Бой реализован в пошаговом режиме. Игрок может:
\begin{vvsu_itemize}
    \item Атаковать (тип атаки зависит от оружия: физическая, пронзающая, магическая).
    \item Использовать предмет из инвентаря.
    \item Попытаться уклониться (временно увеличивает шанс уклонения).
\end{vvsu_itemize}

Система учитывает критические удары, уклонение и защиту. Магический урон частично игнорирует магическую защиту цели.

% Выполнение работы
\section{Выполнение работы}

\subsection{Реализация базовых сущностей}

Базовый класс \texttt{Entity} инкапсулирует общие характеристики всех игровых сущностей. Использование свойств (\texttt{@property}) позволяет динамически рассчитывать итоговые значения с учётом экипировки.

\begin{vvsu_figure}{Базовый класс Entity}{fig:entity}
\begin{minipage}{0.85\linewidth}
\lstinputlisting[language=Python, basicstyle=\fontsize{8}{8}\linespread{1}\selectfont\ttfamily]{../characters/char_classes.py}
\end{minipage}
\end{vvsu_figure}

\subsection{Система классов героя}

Каждый класс героя переопределяет метод \texttt{apply\_stats\_grow}, задавая уникальные правила роста характеристик при повышении уровня. Это демонстрирует полиморфизм.

\begin{vvsu_figure}{Класс Воин (Melee)}{fig:melee}
\begin{minipage}{0.85\linewidth}
\lstinputlisting[firstline=120, lastline=145, language=Python, basicstyle=\fontsize{8}{8}\linespread{1}\selectfont\ttfamily]{../characters/char_classes.py}
\end{minipage}
\end{vvsu_figure}

\subsection{Игровой цикл и подземелье}

Основной игровой цикл реализован в \texttt{game\_flow.py}. Он обрабатывает ввод пользователя, вызывает генерацию комнат и делегирует выполнение событий соответствующим модулям.

\begin{vvsu_figure}{Основной игровой цикл}{fig:gameflow}
\begin{minipage}{0.85\linewidth}
\lstinputlisting[firstline=25, lastline=65, language=Python, basicstyle=\fontsize{8}{8}\linespread{1}\selectfont\ttfamily]{../game/game_flow.py}
\end{minipage}
\end{vvsu_figure}

\subsection{Сохранение и загрузка}

Состояние игры сериализуется в JSON-файл. Для восстановления объектов используется карта соответствия имён предметов их экземплярам (\texttt{ITEM\_MAP}).

\begin{vvsu_figure}{Сохранение игры}{fig:save}
\begin{minipage}{0.85\linewidth}
\lstinputlisting[firstline=70, lastline=100, language=Python, basicstyle=\fontsize{8}{8}\linespread{1}\selectfont\ttfamily]{../persistence/save_load.py}
\end{minipage}
\end{vvsu_figure}

% Тестирование
\section{Тестирование}

Тестирование проводилось вручную в следующих сценариях:
\begin{vvsu_itemize}
    \item Создание персонажа каждого класса и проверка стартовых характеристик.
    \item Прохождение нескольких этажей подземелья с различными комбинациями комнат.
    \item Проверка боевой системы: урон, уклонение, критические удары.
    \item Использование инвентаря: экипировка, использование зелий, выбрасывание предметов.
    \item Сохранение и загрузка игры в разных состояниях (в бою, в меню, после босса).
\end{vvsu_itemize}

В ходе тестирования были выявлены и устранены следующие проблемы:
\begin{vvsu_itemize}
    \item Ошибка при выбрасывании предмета во время боя (исправлена путём блокировки действия).
    \item Некорректное отображение имени героя при загрузке (исправлено в \texttt{get\_save\_info}).
\end{vvsu_itemize}

Игра полностью соответствует требованиям технического задания и демонстрирует работоспособность всех заявленных механик.

\end{document}