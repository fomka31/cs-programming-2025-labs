\documentclass[]{vvsu}

\vvsuyear{2025}

%%%%%%%%%%%%%%%%%%%

\usepackage{graphicx} % для изображений
\usepackage{tabularray} % для таблиц
\usepackage{siunitx} % для обозначений (процент, градус)
\usepackage{listings} % для листингов кода

% Список путей, где будут искаться изображения и файлы
\graphicspath{{images/}}

% Автор документа
\author{Ф.Р.Кучерчук}

% Настройка стилей для листингов кода
\input{listing_styles.tex}

%%%%%%%%%%%%%%%%%%%

\begin{document}

% Шапка
\vvsuhead{\linespread{1}\selectfont{}МИНОБРНАУКИ РОССИИ\\
\vspace{10pt}Федеральное государственное бюджетное образовательное учреждение\\
высшего образования\\
\fontsize{13}{13}\selectfont{}<<ВЛАДИВОСТОКСКИЙ ГОСУДАРСТВЕННЫЙ УНИВЕРСИТЕТ>>\\
(ФГБОУ ВО <<ВВГУ>>)\\
\vspace{10pt}\fontsize{12}{12}\selectfont{}ИНСТИТУТ ИНФОРМАЦИОННЫХ ТЕХНОЛОГИЙ И АНАЛИЗА ДАННЫХ\\
КАФЕДРА ИНФОРМАЦИОННЫХ ТЕХНОЛОГИЙ И СИСТЕМ}

% Название отчета
\title{Отчет\\по лабораторной работе №5}
\subtitle{по дисциплине\\<<Информатика и программирование>>}

% Участники работы
\member{Студент\\ гр. БИН-25-2}{К.Ф. Кучерчук}
\member{Ассистент\\ преподавателя}{М.В. Водяницкий}

% Вывод титульника
\maketitle

% Задание
\begin{addition}{Задание}
  Выполнить задания и оформить отчет по стандартам ВВГУ.

   \textit{\textbf{Задание 1.}}  
  Дан список из 10 различных целых чисел. Необходимо найти в нем число 3 и заменить на 30.

  \textit{\textbf{Задание 2.}}  
  Дан список из 5 целых чисел. Необходимо превратить его в список квадратов этих чисел.

  \textit{\textbf{Задание 3.}}  
  Имеется список различных целых чисел. Программа должна найти наибольшее из чисел списка и разделить его на длину списка.

  \textit{\textbf{Задание 4.}}  
  Имеется кортеж из нескольких произвольных элементов. Необходимо этот кортеж отсортировать. Если хотя бы один элемент не является числом, то кортеж остается неизменным.

  \textit{\textbf{Задание 5.}}  
  Имеется словарь товаров в магазине. Необходимо найти товар с минимальной и максимальной ценой.

  \textit{\textbf{Задание 6.}}  
  Имеется список произвольных элементов. Необходимо на основе этого списка создать словарь, где каждый элемент списка будет и ключом, и значением.

  \textit{\textbf{Задание 7.}}  
  Имеется словарь перевода английских слов на русский, где ключ английского слово, значение - русского. Необходимо реализовать программу которая получает на ввод русское слово и результатом выдает перевод на английский.

  \textit{\textbf{Задание 8.}}  
  Реализовать игру Камень-Ножницы-Бумага-Ящерица-Спок. Программа должна запрашивать у пользователя ввод одного из вариантов. Второй вариант случайно генерирует сама программа и возвращает победителя.

  Пример:\\
  \begin{vvsu_itemize}
    \item Ножницы режут бумагу
    \item Бумага покрывает камень
    \item Камень давит ящерицу
    \item Ящерица отравляет Спока
    \item Спок ломает ножницы
    \item Ножницы обезглавливают ящерицу
    \item Ящерица съедает бумагу
    \item Бумага подставляет Спока
    \item Спок испаряет камень
    \item Камень разбивает ножницы
  \end{vvsu_itemize}
  
  \textit{\textbf{Задание 9.}}  
  Дан список слов - например:\\

  `["яблоко", "груша", "банан", "киви", "апельсин", "ананас"]`\\

  Необходимо создать новый словарь, где:

  \begin{vvsu_itemize}
    \item Ключом будет первая буква слова
    \item Значением - список всех слов, начинающихся с этой буквы
  \end{vvsu_itemize}
  
  Пример результата:\\

  {'я': ['яблоко'], 'г': ['груша'], 'б': ['банан'], 'к': ['киви'], 'а': ['апельсин', 'ананас']}\\

  \textit{\textbf{Задание 10.}}  
  Дан список кортежей, где каждый кортеж содержит имя студента и его оценки, например:\\
  
  [("Анна", [5, 4, 5]), ("Иван", [3, 4, 4]), ("Мария", [5, 5, 5])]\\

  Необходимо:

  \begin{enumerate}
    \item Создать словарь, где ключ - имя студента, значение - его средняя оценка
    \item Найти студента с наибольшей средней оценкой и вывести его имя и средний балл
  \end{enumerate}

  Пример результата:\\
  Мария имеет наивысший средний балл: 5.0

\end{addition}


% Содержание
\toc

% Глава - Выполнение работы
\section{Выполнение работы}

% Подглава - Задание 1
\subsection{Задание 1}

Создаём список lst, который наполняем числами. После при помощи цикла for проходимся по всем элементам, если этот элемент, переведённый в строку функцией str(), совпадает со строкой '3', то мы этому элементу присваиваем значение 30. На рисунке \ref{fig:code_task_1} представлен код программы.

\begin{vvsu_figure}{Листинг программы для задания 1}{fig:code_task_1}
  \begin{minipage}{.75\textwidth}
    \lstinputlisting[language=Python,basicstyle=\fontsize{10}{10}\linespread{1}\selectfont\ttfamily]{code/task1.py}
  \end{minipage}
\end{vvsu_figure}

% Подглава - Задание 2
\subsection{Задание 2}

Создаём список lst, который наполняем числами. При помощи функции map(), первым аргументом в который мы передаём лямбда-функцию, которая возводит аргумент в квадрат, а вторым сам список чисел, и оборачиваем в список, затем присваиваем его нашему первоначальному списку. На рисунке \ref{fig:code_task_2} представлен код программы.

\begin{vvsu_figure}{Листинг программы для задания 2}{fig:code_task_2}
  \begin{minipage}{.75\textwidth}
    \lstinputlisting[language=Python,basicstyle=\fontsize{10}{10}\linespread{1}\selectfont\ttfamily]{code/task2.py}
  \end{minipage}
\end{vvsu_figure}

% Подглава - Задание 3
\subsection{Задание 3}

Создаём список lst, который наполняем числами. Переменной result присваиваем число с точкой, полученное делением наибольшего числа из списка, полученного функцией max() со списком в качестве аргумента, на длинну списка, полученной при помощи функции len() со списком в качестве аргумента. На рисунке \ref{fig:code_task_3} представлен код программы.

\begin{vvsu_figure}{Листинг программы для задания 3}{fig:code_task_3}
  \begin{minipage}{.75\textwidth}
    \lstinputlisting[language=Python,basicstyle=\fontsize{10}{10}\linespread{1}\selectfont\ttfamily]{code/task3.py}
  \end{minipage}
\end{vvsu_figure}

% Подглава - Задание 4
\subsection{Задание 4}

Создаём кортеж чисел, присваивая его переменной tpl. Создаём новую переменную tpl1, в который заносим строку, полученную следующим образом: метод join применяется к строке и вставляет её между элементами итерируемого объекта со строками, который мы получаем применяя функцию map к кортежу tpl и функцией str(). Далее переменной tpl тернарным оператором сортированный кортеж tpl(при помощи функции sorted()) если все элементы строки tpl1 цифры(при помощи метода isdigit()), в противном случае оставляем кортеж неизменным. На рисунке \ref{fig:code_task_4} представлен код решения.

\begin{vvsu_figure}{Листинг программы для задания 4}{fig:code_task_4}
  \begin{minipage}{.75\textwidth}
    \lstinputlisting[language=Python,basicstyle=\fontsize{10}{10}\linespread{1}\selectfont\ttfamily]{code/task4.py}
  \end{minipage}
\end{vvsu_figure}


% Подглава - Задание 5
\subsection{Задание 5}

Создаём словарь pricelist, ключами которого являются наименования таваров, а значениями их стоимость. Далее создаём переменные maximum и minimum, разницей в которых будет только применяемая функция, min() для минимального и max() для максимального. Первым аргументом передаём элементы словаря при помощи метода items(), а вторым ключ для функции, для того чтобы сортировка была по значению а не по ключу. Затем выводим это пользователю функцией print(). На рисунке \ref{fig:code_task_5} представлен код программы.

\begin{vvsu_figure}{Листинг программы для задания 5}{fig:code_task_5}
  \begin{minipage}{.75\textwidth}
    \lstinputlisting[language=Python,basicstyle=\fontsize{10}{10}\linespread{1}\selectfont\ttfamily]{code/task5.py}
  \end{minipage}
\end{vvsu_figure}

% Подглава - Задание 6
\subsection{Задание 6}

Список lst содержит элементы данных произвольных типов. Создаём словарь dct. Дальше циклом for проходимся по всем элементам списка. Внутри него открывается блок try expect для отлова ошибок. Ошибка же может быть при попытке создания элемента словаря с ключом i и значением i, так как существуют ограничения для его ключей. Если ошибка возникает, то мы просто идём на следующую итерацию. На рисунке \ref{fig:code_task_6} представлен код программы.

\begin{vvsu_figure}{Листинг программы для задания 6}{fig:code_task_6}
  \begin{minipage}{.75\textwidth}
    \lstinputlisting[language=Python,basicstyle=\fontsize{10}{10}\linespread{1}\selectfont\ttfamily]{code/task6.py}
  \end{minipage}
\end{vvsu_figure}

% Подглава - Задание 7
\subsection{Задание 7}

Создаём словарь trans_dict, в котором в качестве ключей используются английские слова, а в качестве значений соответствующие им переводы. Далее при помощи функции input запрашиваем у пользователя строку на ввод и присваиваем её переменной word. Далее в блоке try expect пробуем обратиться к элементу словаря с ключом word и вывести информацию, в случае ошибки выводится сообщение о том, что элемента нет в словаре. На рисунке \ref{fig:code_task_7} представлен код программы.

\begin{vvsu_figure}{Листинг программы для задания 7}{fig:code_task_7}
  \begin{minipage}{.75\textwidth}
    \lstinputlisting[language=Python,basicstyle=\fontsize{10}{10}\linespread{1}\selectfont\ttfamily]{code/task7.py}
  \end{minipage}
\end{vvsu_figure}

% Подглава - Задание 8
\subsection{Задание 8}

Импортируем модуль random, который позволяет компьютеру делать случайный выбор. Создает словарь с правилами игры. Каждый ключ (например, 'ножницы') содержит список того, что он побеждает. Создаем список choices всех возможных выборов ['ножницы', 'бумага', 'камень', 'ящерица', 'спок']. rules.keys() берет все ключи из словаря rules. Создаём функцию print_choices для вывода выборов игрока и компьютера.global означает, что функция использует переменные, созданные вне ее. Далее Компьютер случайно выбирает один из вариантов (ножницы/бумага/камень/ящерица/спок) функцией choice модуля random. Потом показываем пользователю список доступных знаков. Запрашиваем у пользователя строку на ввод в переменную player_choice. Далее в блоке try expect ищем выборы пользоателя и ИИ в словаре и выводим на жкран сообщение о результате, предварительно вызвав функцию print_choices. Если выбора игрока нету в ключах словаря, то выводим сообщение об этом на экран. На рисунке \ref{fig:code_task_8} представлен код программы.

\begin{vvsu_figure}{Листинг программы для задания 8}{fig:code_task_8}
  \begin{minipage}{.75\textwidth}
    \lstinputlisting[language=Python,basicstyle=\fontsize{10}{10}\linespread{1}\selectfont\ttfamily]{code/task8.py}
  \end{minipage}
\end{vvsu_figure}

% Подглава - Задание 9
\subsection{Задание 9}

. На рисунке \ref{fig:code_task_9} представлен код программы.

\begin{vvsu_figure}{Листинг программы для задания 9}{fig:code_task_9}
  \begin{minipage}{.75\textwidth}
    \lstinputlisting[language=Python,basicstyle=\fontsize{10}{10}\linespread{1}\selectfont\ttfamily]{code/task9.py}
  \end{minipage}
\end{vvsu_figure}

% Подглава - Задание 10
\subsection{Задание 10}

Получаем число от пользователя с помощью input(), int() преобразовывает строку в целое число. При вводе нечисловых данных выводим ошибку.
После того, как мы убедились в корректности данных, начинаем обрабатывать число, а именно: если число <  1 или четное, то оно уже не простое, кроме 2. Далее запускаем цикл, в котором мы будем искать делители нашего числа, если найдутся еще кроме 1 и самого числа, то выводим в консоль, что число составное, в ином случае - простое. На рисунке \ref{fig:code_task_10} представлен код программы.

\begin{vvsu_figure}{Листинг программы для задания 10}{fig:code_task_10}
  \begin{minipage}{.75\textwidth}
    \lstinputlisting[language=Python,basicstyle=\fontsize{10}{10}\linespread{1}\selectfont\ttfamily]{code/task10.py}
  \end{minipage}
\end{vvsu_figure}

Спасибо за внимание !

\end{document}