\documentclass[]{vvsu}

\vvsuyear{2025}

%%%%%%%%%%%%%%%%%%%

\usepackage{graphicx}
\usepackage{tabularray}
\usepackage{siunitx}
\usepackage{listings}
\usepackage[utf8]{inputenc}
\usepackage[russian]{babel}

\graphicspath{{images/}}

\author{К.Ф. Кучерчук}

\input{listing_styles.tex}

%%%%%%%%%%%%%%%%%%%

\begin{document}

% Шапка
\vvsuhead{\linespread{1}\selectfont{}МИНОБРНАУКИ РОССИИ\\
\vspace{10pt}Федеральное государственное бюджетное образовательное учреждение\\
высшего образования\\
\fontsize{13}{13}\selectfont{}<<ВЛАДИВОСТОКСКИЙ ГОСУДАРСТВЕННЫЙ УНИВЕРСИТЕТ>>\\
(ФГБОУ ВО <<ВВГУ>>)\\
\vspace{10pt}\fontsize{12}{12}\selectfont{}ИНСТИТУТ ИНФОРМАЦИОННЫХ ТЕХНОЛОГИЙ И АНАЛИЗА ДАННЫХ\\
КАФЕДРА ИНФОРМАЦИОННЫХ ТЕХНОЛОГИЙ И СИСТЕМ}

% Название отчета
\title{Отчет\\по лабораторной работе №6}
\subtitle{по дисциплине\\<<Информатика и программирование>>}

% Участники работы
\member{Студент\\ гр. БИН-25-2}{К.Ф. Кучерчук}
\member{Ассистент\\ преподавателя}{М.В. Водяницкий}

% Вывод титульника
\maketitle

% Задание
\begin{addition}{Задание}
  Выполнить задания и оформить отчет по стандартам ВВГУ.

  \textbf{Основные задания}

  \vspace{6pt}
  \textbf{Задание 1}

  Написать функцию, которая конвертирует время из одной величины в другую.

  На вход подается:
  \begin{vvsu_itemize}
    \item число (величина времени)
    \item исходная единица измерения
    \item единица измерения, в которую нужно перевести
  \end{vvsu_itemize}

  Функция должна вернуть конвертированное значение.

  Примеры (формат ввода/вывода можно выбрать свой, если нет строгих требований):

  \begin{tabular}{|c|c|}
    \hline
    Вход & Выход \\
    \hline
    \texttt{4h m} & \texttt{240m} \\
    \texttt{30m h} & \texttt{0.5h} \\
    \texttt{12s h} & \texttt{0.03h} \\
    \hline
  \end{tabular}

  \vspace{12pt}
  \textbf{Задание 2}

  Пользователь делает вклад в банке в размере \texttt{a} рублей сроком на \texttt{n} лет. Процент по вкладу \textbf{зависит от суммы и срока}.

  \textbf{Зависимость от суммы:}
  \begin{vvsu_itemize}
    \item каждые 10\,000 рублей увеличивают ставку на 0.3\%
    \item но суммарное увеличение не может превышать 5\%
    \item минимальный вклад — 30\,000 рублей
  \end{vvsu_itemize}

  \textbf{Зависимость от срока:}
  \begin{vvsu_itemize}
    \item первые 3 года — 3\%
    \item от 4 до 6 лет — 5\%
    \item более 6 лет — 2\%
  \end{vvsu_itemize}

  Необходимо написать функцию, которая рассчитывает прибыль пользователя без учета первоначально вложенной суммы. Используется сложный процент: каждый год процент начисляется на текущую сумму вклада.

  На вход подаются: сумма вклада и количество лет. Результат: сумма прибыли (не весь вклад, а только заработанные проценты).

  Примеры:

  \begin{tabular}{|c|c|}
    \hline
    Вход & Выход \\
    \hline
    \texttt{30000 3} & \texttt{3648.67} \\
    \texttt{100000 5} & \texttt{38920.10} \\
    \texttt{200000 8} & \texttt{183925.42} \\
    \hline
  \end{tabular}

  \vspace{12pt}
  \textbf{Задание 3}

  Написать функцию для вывода всех простых чисел в заданном диапазоне. Нужно учитывать некорректные данные (например, начало больше конца или диапазон без простых чисел).

  На вход подаются два числа: начало и конец диапазона (включительно). На выходе — список всех простых чисел или сообщение об ошибке.

  Примеры:

  \begin{tabular}{|c|c|}
    \hline
    Вход & Выход \\
    \hline
    \texttt{1 10} & \texttt{2 3 5 7} \\
    \texttt{15 120} & \texttt{17 19 23 29 31 37 41 43 47 53 59 61 67 71 73 79 83 89 97 101 103 107 109 113} \\
    \texttt{0 1} & \texttt{Error!} \\
    \hline
  \end{tabular}

  \vspace{12pt}
  \textbf{Задание 4}

  Реализовать функцию сложения двух матриц.

  При сложении двух матриц получается новая матрица того же размера, где каждый элемент — это сумма элементов с тем же индексом из двух исходных матриц.

  Ограничения:
  \begin{vvsu_itemize}
    \item складывать можно только матрицы одинакового размера
    \item размер матрицы должен быть строго больше 2 (например, 3×3, 4×4 и т.д.)
    \item при нарушении условий нужно вывести сообщение об ошибке
  \end{vvsu_itemize}

  На вход подаются:
  \begin{enumerate}
    \item размер матрицы \texttt{n} (для квадратной матрицы \texttt{n × n})
    \item элементы первой матрицы (по строкам, через пробел)
    \item элементы второй матрицы в таком же формате
  \end{enumerate}

  Результат — новая матрица (в том же формате), либо сообщение об ошибке.

  Пример (один из возможных вариантов формата):

  Вход:
  \begin{lstlisting}[basicstyle=\ttfamily\small]
2
2 5
5 3
5 2
4 1
  \end{lstlisting}

  Выход:
  \begin{lstlisting}[basicstyle=\ttfamily\small]
7 7
9 4
  \end{lstlisting}

  Пример с ошибкой (слишком маленький размер, неправильный ввод и т.п.):

  Вход:
  \begin{lstlisting}[basicstyle=\ttfamily\small]
1
4
5
  \end{lstlisting}

  Выход:
  \begin{lstlisting}[basicstyle=\ttfamily\small]
Error!
  \end{lstlisting}

  \vspace{12pt}
  \textbf{Задание 5}

  Написать функцию, которая определяет, является ли строка палиндромом.

  Палиндром — это строка, которая читается одинаково слева направо и справа налево (обычно без учета пробелов, регистра и знаков препинания — эти правила нужно явно задать в своей реализации).

  На вход подается строка. На выходе:
  \begin{vvsu_itemize}
    \item \texttt{Да}, если это палиндром
    \item \texttt{Нет}, если это не палиндром
  \end{vvsu_itemize}

  Примеры:

  \begin{tabular}{|c|c|}
    \hline
    Вход & Выход \\
    \hline
    \texttt{А роза упала на лапу Азора} & \texttt{Да} \\
    \texttt{Borrow or rob} & \texttt{Да} \\
    \texttt{Алфавитный порядок} & \texttt{Нет} \\
    \hline
  \end{tabular}
\end{addition}

% Содержание
\toc

% Глава - Выполнение работы
\section{Выполнение работы}

\subsection{Задание 1}

Программа реализует функцию преобразования единиц времени (секунды, минуты, часы, дни). На вход подаётся строка вида «12s h», где число и исходная единица отделяются от целевой единицы пробелом. Функция использует словарь для хранения коэффициентов перевода в секунды.

\begin{vvsu_figure}{Листинг программы для задания 1}{fig:code_task_1}
  \begin{minipage}{.75\textwidth}
    \lstinputlisting[language=Python,basicstyle=\fontsize{10}{10}\linespread{1}\selectfont\ttfamily]{code/task1.py}
  \end{minipage}
\end{vvsu_figure}

Внутри функции строка разбивается на исходное значение и целевую единицу. Числовая часть извлекается путём удаления последнего символа (обозначения единицы), а коэффициенты из словаря используются для перевода в целевую систему. Результат округляется до трёх знаков после запятой и возвращается в виде строки с суффиксом целевой единицы.

\subsection{Задание 2}

Функция рассчитывает прибыль по вкладу с учётом сложного процента. Она учитывает минимальную сумму вклада (30 000 руб.), бонусную ставку (до 5\% в зависимости от суммы) и основную ставку (в зависимости от срока). Прибыль вычисляется как разница между итоговой суммой и первоначальным вкладом.

\begin{vvsu_figure}{Листинг программы для задания 2}{fig:code_task_2}
  \begin{minipage}{.75\textwidth}
    \lstinputlisting[language=Python,basicstyle=\fontsize{10}{10}\linespread{1}\selectfont\ttfamily]{code/task2.py}
  \end{minipage}
\end{vvsu_figure}

Сначала проверяются ограничения: сумма не менее 30 000, срок — не менее 1 года. Затем вычисляется бонус: 0.3\% за каждые 10 000 руб., но не более 5\%. Основная ставка определяется по диапазону срока. Формула сложного процента применяется единожды: \( S = P \cdot (1 + r)^n - P \). Однако в текущей реализации допущена ошибка: условие `4 < srok <= 7` не охватывает 4 года. Это следует исправить на `srok <= 3`, `4 <= srok <= 6`, иначе `else`.

\subsection{Задание 3}

Функция принимает два целых числа — границы диапазона — и возвращает все простые числа в нём. Если простых чисел нет, возвращается строка «Error!». Диапазон всегда начинается с \texttt{max(2, a)}, так как числа меньше 2 не могут быть простыми.

\begin{vvsu_figure}{Листинг программы для задания 3}{fig:code_task_3}
  \begin{minipage}{.75\textwidth}
    \lstinputlisting[language=Python,basicstyle=\fontsize{10}{10}\linespread{1}\selectfont\ttfamily]{code/task3.py}
  \end{minipage}
\end{vvsu_figure}

Для каждого числа от \texttt{max(2, a)} до \texttt{b} проверяется делимость на все целые от 2 до \(\sqrt{n}\). Если делитель найден — число составное. Иначе — простое и добавляется в список. Это классический метод проверки простоты «перебором до корня».

\subsection{Задание 4}

Программа складывает две квадратные матрицы одинакового размера. Перед началом проверяется, что размер матрицы строго больше 2. Если нет — выводится «Error!». При любых ошибках ввода (неверное число элементов, нечисловые значения и т.п.) также выводится сообщение об ошибке.

\begin{vvsu_figure}{Листинг программы для задания 4}{fig:code_task_4}
  \begin{minipage}{.75\textwidth}
    \lstinputlisting[language=Python,basicstyle=\fontsize{10}{10}\linespread{1}\selectfont\ttfamily]{code/task4.py}
  \end{minipage}
\end{vvsu_figure}

Чтение матриц осуществляется построчно с помощью генератора списков и \texttt{map(int, input().split())}. Сложение выполняется поэлементно в цикле. Обработка ошибок реализована через универсальный блок \texttt{except}, что соответствует требованиям задания.

\subsection{Задание 5}

Программа определяет, является ли введённая строка палиндромом. Для этого удаляются все неалфавитно-цифровые символы, строка приводится к нижнему регистру, и сравнивается с обратной копией.

\begin{vvsu_figure}{Листинг программы для задания 5}{fig:code_task_5}
  \begin{minipage}{.75\textwidth}
    \lstinputlisting[language=Python,basicstyle=\fontsize{10}{10}\linespread{1}\selectfont\ttfamily]{code/task5.py}
  \end{minipage}
\end{vvsu_figure}

Очистка строки реализована с помощью генератора: `c.isalnum()` отбирает только буквы и цифры. Метод `[::-1]` создаёт реверс строки. Сравнение даёт булев результат, который преобразуется в «Да» или «Нет» через тернарный оператор. Такой подход корректно обрабатывает фразы на русском и английском, игнорируя пробелы и пунктуацию.

\end{document}