\documentclass[]{vvsu}

\vvsuyear{2025}

%%%%%%%%%%%%%%%%%%%

\usepackage{graphicx}
\usepackage{tabularray}
\usepackage{siunitx}
\usepackage{listings}
\usepackage[utf8]{inputenc}
\usepackage[russian]{babel}

\graphicspath{{images/}}

\author{К.Ф. Кучерчук}

\input{listing_styles.tex}

%%%%%%%%%%%%%%%%%%%

\begin{document}

% Шапка
\vvsuhead{\linespread{1}\selectfont{}МИНОБРНАУКИ РОССИИ\\
\vspace{10pt}Федеральное государственное бюджетное образовательное учреждение\\
высшего образования\\
\fontsize{13}{13}\selectfont{}<<ВЛАДИВОСТОКСКИЙ ГОСУДАРСТВЕННЫЙ УНИВЕРСИТЕТ>>\\
(ФГБОУ ВО <<ВВГУ>>)\\
\vspace{10pt}\fontsize{12}{12}\selectfont{}ИНСТИТУТ ИНФОРМАЦИОННЫХ ТЕХНОЛОГИЙ И АНАЛИЗА ДАННЫХ\\
КАФЕДРА ИНФОРМАЦИОННЫХ ТЕХНОЛОГИЙ И СИСТЕМ}

% Название отчета
\title{Отчет\\по лабораторной работе №7}
\subtitle{по дисциплине\\<<Информатика и программирование>>}

% Участники работы
\member{Студент\\ гр. БИН-25-2}{К.Ф. Кучерчук}
\member{Ассистент\\ преподавателя}{М.В. Водяницкий}

% Вывод титульника
\maketitle

% Задание
\begin{addition}{Задание}
  Выполнить задания и оформить отчет по стандартам ВВГУ.

  \textbf{Основные задания}

  \vspace{6pt}
  \textbf{Задание 1}

  Имеется список объектов Фонда с указанием уровня угрозы:

  \begin{lstlisting}[basicstyle=\ttfamily\small]
objects = [
    ("Containment Cell A", 4),
    ("Archive Vault", 1),
    ("Bio Lab Sector", 3),
    ("Observation Wing", 2)
]
  \end{lstlisting}

  Используя \texttt{sorted} и лямбда-выражение, отсортируйте объекты по возрастанию уровня угрозы.

  \vspace{12pt}
  \textbf{Задание 2}

  Дан список сотрудников Фонда с количеством проведенных смен и стоимостью одной смены:

  \begin{lstlisting}[basicstyle=\ttfamily\small]
staff_shifts = [
    {"name": "Dr. Shaw", "shift_cost": 120, "shifts": 15},
    {"name": "Agent Torres", "shift_cost": 90, "shifts": 22},
    {"name": "Researcher Hall", "shift_cost": 150, "shifts": 10}
]
  \end{lstlisting}

  Используя \texttt{map} и лямбда-выражение, создайте список общей стоимости работы каждого сотрудника.  
  Затем найдите максимальную стоимость с помощью \texttt{max}.

  \vspace{12pt}
  \textbf{Задание 3}

  Дан список персонала с уровнем допуска:

  \begin{lstlisting}[basicstyle=\ttfamily\small]
personnel = [
    {"name": "Dr. Klein", "clearance": 2},
    {"name": "Agent Brooks", "clearance": 4},
    {"name": "Technician Reed", "clearance": 1}
]
  \end{lstlisting}

  Используя \texttt{map} и лямбда-выражение, создайте новый список, где каждому сотруднику добавляется категория допуска:
  \begin{vvsu_itemize}
    \item \texttt{"Restricted"} — уровень 1
    \item \texttt{"Confidential"} — уровни 2–3
    \item \texttt{"Top Secret"} — уровень 4 и выше
  \end{vvsu_itemize}
  Результат должен быть списком словарей.

  \vspace{12pt}
  \textbf{Задание 4}

  Дан список зон Фонда с указанием времени активности (в часах):

  \begin{lstlisting}[basicstyle=\ttfamily\small]
zones = [
    {"zone": "Sector-12", "active_from": 8, "active_to": 18},
    {"zone": "Deep Storage", "active_from": 0, "active_to": 24},
    {"zone": "Research Wing", "active_from": 9, "active_to": 17}
]
  \end{lstlisting}

  Используя \texttt{filter} и лямбда-выражение, выберите зоны, которые полностью работают в дневной период (с 8 до 18 включительно).

  \vspace{12pt}
  \textbf{Задание 5}

  Фонд анализирует служебные отчеты. Некоторые отчеты содержат внешние ссылки, которые должны быть удалены перед архивированием.

  Приведён список из 15 отчётов (см. исходный файл).  
  Используя \texttt{filter} и лямбда-выражение:
  \begin{enumerate}
    \item Отберите отчёты, содержащие ссылки (\texttt{http} или \texttt{https})
    \item Преобразуйте их так, чтобы вместо ссылки отображалось \texttt{[ДАННЫЕ УДАЛЕНЫ]}
  \end{enumerate}

  \vspace{12pt}
  \textbf{Задание 6}

  Дан список SCP-объектов с указанием их класса содержания:

  \begin{lstlisting}[basicstyle=\ttfamily\small]
scp_objects = [
    {"scp": "SCP-096", "class": "Euclid"},
    {"scp": "SCP-173", "class": "Euclid"},
    {"scp": "SCP-055", "class": "Keter"},
    {"scp": "SCP-999", "class": "Safe"},
    {"scp": "SCP-3001", "class": "Keter"}
]
  \end{lstlisting}

  Используя \texttt{filter} и лямбда-выражение, сформируйте список SCP-объектов, которые требуют усиленных мер содержания.  
  К объектам с усиленными мерами относятся все SCP, \textbf{класс которых не равен \texttt{"Safe"}}.  
  Результат должен быть списком словарей исходного формата.

  \vspace{12pt}
  \textbf{Задание 7}

  Дан список инцидентов с количеством задействованного персонала:

  \begin{lstlisting}[basicstyle=\ttfamily\small]
incidents = [
    {"id": 101, "staff": 4},
    {"id": 102, "staff": 12},
    {"id": 103, "staff": 7},
    {"id": 104, "staff": 20}
]
  \end{lstlisting}

  Используя \texttt{sorted} и лямбда-выражение:
  \begin{enumerate}
    \item Отсортируйте инциденты по количеству персонала
    \item Оставьте только три наиболее ресурсоемких инцидента
  \end{enumerate}

  \vspace{12pt}
  \textbf{Задание 8}

  Дан список протоколов безопасности и их уровней критичности:

  \begin{lstlisting}[basicstyle=\ttfamily\small]
protocols = [
    ("Lockdown", 5),
    ("Evacuation", 4),
    ("Data Wipe", 3),
    ("Routine Scan", 1)
]
  \end{lstlisting}

  Используя \texttt{map} и лямбда-выражение, создайте новый список строк вида:  
  \texttt{"Protocol Lockdown - Criticality 5"}

  \vspace{12pt}
  \textbf{Задание 9}

  Имеется список смен охраны с указанием длительности (в часах):

  \begin{lstlisting}[basicstyle=\ttfamily\small]
shifts = [6, 12, 8, 24, 10, 4]
  \end{lstlisting}

  Используя \texttt{filter} и лямбда-выражение, выберите только те смены, которые:
  \begin{vvsu_itemize}
    \item длятся не менее 8 часов
    \item не превышают 12 часов
  \end{vvsu_itemize}

  \vspace{12pt}
  \textbf{Задание 10}

  Дан список сотрудников с результатами психологической оценки (от 0 до 100):

  \begin{lstlisting}[basicstyle=\ttfamily\small]
evaluations = [
    {"name": "Agent Cole", "score": 78},
    {"name": "Dr. Weiss", "score": 92},
    {"name": "Technician Moore", "score": 61},
    {"name": "Researcher Lin", "score": 88}
]
  \end{lstlisting}

  Используя \texttt{max} и лямбда-выражение, определите сотрудника с наивысшей оценкой.  
  Результатом должно быть имя сотрудника и его балл.
\end{addition}

% Содержание
\toc

% Глава - Выполнение работы
\section{Выполнение работы}

\subsection{Задание 1}

Требуется отсортировать список кортежей, представляющих объекты Фонда, по возрастанию уровня угрозы (второй элемент кортежа). Для этого используется встроенная функция \texttt{sorted} с ключом сортировки, заданным лямбда-выражением.

\begin{vvsu_figure}{Листинг программы для задания 1}{fig:code_task_1}
  \begin{minipage}{.75\textwidth}
    \lstinputlisting[language=Python,basicstyle=\fontsize{10}{10}\linespread{1}\selectfont\ttfamily]{code/task1.py}
  \end{minipage}
\end{vvsu_figure}

Функция \texttt{sorted} принимает список \texttt{objects} и ключ сортировки \texttt{key=lambda x: x[1]}, который извлекает второй элемент каждого кортежа. Параметр \texttt{reverse=False} (по умолчанию) обеспечивает сортировку по возрастанию. Результат — новый список объектов, упорядоченный от наименее угрожающего к наиболее угрожающему.

\subsection{Задание 2}

Необходимо рассчитать общую стоимость работы каждого сотрудника как произведение количества смен на стоимость одной смены, а затем найти максимальное значение среди всех сотрудников.

\begin{vvsu_figure}{Листинг программы для задания 2}{fig:code_task_2}
  \begin{minipage}{.75\textwidth}
    \lstinputlisting[language=Python,basicstyle=\fontsize{10}{10}\linespread{1}\selectfont\ttfamily]{code/task2.py}
  \end{minipage}
\end{vvsu_figure}

С помощью \texttt{map} и лямбда-функции \texttt{lambda x: x['shift\_cost'] * x['shifts']} вычисляется общая стоимость для каждого сотрудника. Результат преобразуется в список \texttt{total\_shifts}. Затем функция \texttt{max} находит наибольшее значение в этом списке. Отметим, что в текущей реализации возвращается только максимальная сумма, без указания, кому она принадлежит.

\subsection{Задание 3}

Требуется преобразовать числовые значения уровня допуска в строковые категории: «Restricted», «Confidential» или «Top Secret», согласно заданному отображению.

\begin{vvsu_figure}{Листинг программы для задания 3}{fig:code_task_3}
  \begin{minipage}{.75\textwidth}
    \lstinputlisting[language=Python,basicstyle=\fontsize{10}{10}\linespread{1}\selectfont\ttfamily]{code/task3.py}
  \end{minipage}
\end{vvsu_figure}

Сначала создаётся словарь \texttt{clearances}, задающий соответствие уровня допуска его названию. Затем с помощью \texttt{map} и лямбда-выражения \texttt{lambda x: clearances.get(x['clearance'])} формируется список строковых категорий. Далее исходный список словарей обновляется: числовое значение \texttt{clearance} заменяется на строковое. Такой подход изменяет исходные данные на месте.

\subsection{Задание 4}

Необходимо отфильтровать список зон, оставив только те, которые активны в течение всего дневного периода — с 8:00 до 18:00 включительно. Это означает, что зона должна начинать работу не позже 8:00 и заканчивать не раньше 18:00.

\begin{vvsu_figure}{Листинг программы для задания 4}{fig:code_task_4}
  \begin{minipage}{.75\textwidth}
    \lstinputlisting[language=Python,basicstyle=\fontsize{10}{10}\linespread{1}\selectfont\ttfamily]{code/task4.py}
  \end{minipage}
\end{vvsu_figure}

Функция \texttt{filter} применяет лямбда-условие \texttt{x['active\_from'] <= 8 and x['active\_to'] >= 18}, проверяющее полное покрытие дневного интервала. Например, зона «Sector-12» (8–18) проходит фильтр, а «Research Wing» (9–17) — нет. Результат преобразуется в список, содержащий только подходящие зоны.

\subsection{Задание 5}

Требуется извлечь отчёты, содержащие URL-ссылки, и заменить каждую ссылку на строку \texttt{[ДАННЫЕ УДАЛЕНЫ]}, сохранив остальной текст без изменений.

\begin{vvsu_figure}{Листинг программы для задания 5}{fig:code_task_5}
  \begin{minipage}{.75\textwidth}
    \lstinputlisting[language=Python,basicstyle=\fontsize{10}{10}\linespread{1}\selectfont\ttfamily]{code/task5.py}
  \end{minipage}
\end{vvsu_figure}

Сначала \texttt{filter} отбирает отчёты, в тексте которых присутствует \texttt{http://} или \texttt{https://}. Затем определяется вспомогательная функция \texttt{remove\_urls\_from\_text}, которая разбивает текст на слова и заменяет слова, начинающиеся с \texttt{http}, на заглушку. Наконец, \texttt{map} применяет эту функцию ко всем отфильтрованным отчётам, формируя новый список с очищенным текстом.

\subsection{Задание 6}

Следует отфильтровать SCP-объекты, оставив только те, чей класс не равен \texttt{"Safe"}, так как именно они требуют усиленных мер содержания.

\begin{vvsu_figure}{Листинг программы для задания 6}{fig:code_task_6}
  \begin{minipage}{.75\textwidth}
    \lstinputlisting[language=Python,basicstyle=\fontsize{10}{10}\linespread{1}\selectfont\ttfamily]{code/task6.py}
  \end{minipage}
\end{vvsu_figure}

Функция \texttt{filter} с лямбда-условием \texttt{x['class'] != 'Safe'} отбирает объекты классов «Euclid» и «Keter». Результат преобразуется в список, содержащий исходные словари без изменений, что соответствует требованию.

\subsection{Задание 7}

Необходимо отсортировать инциденты по количеству задействованного персонала в порядке убывания и выбрать три первых (наиболее ресурсоёмких).

\begin{vvsu_figure}{Листинг программы для задания 7}{fig:code_task_7}
  \begin{minipage}{.75\textwidth}
    \lstinputlisting[language=Python,basicstyle=\fontsize{10}{10}\linespread{1}\selectfont\ttfamily]{code/task7.py}
  \end{minipage}
\end{vvsu_figure}

Сначала список \texttt{incidents} сортируется по ключу \texttt{x['staff']} с параметром \texttt{reverse=True}, чтобы самые крупные инциденты оказались в начале. Затем берётся срез первых трёх элементов с помощью \texttt{[:3]}. Результат — список из трёх словарей с наибольшим значением \texttt{staff}.

\subsection{Задание 8}

Требуется преобразовать список кортежей (протокол, критичность) в список строк заданного формата.

\begin{vvsu_figure}{Листинг программы для задания 8}{fig:code_task_8}
  \begin{minipage}{.75\textwidth}
    \lstinputlisting[language=Python,basicstyle=\fontsize{10}{10}\linespread{1}\selectfont\ttfamily]{code/task8.py}
  \end{minipage}
\end{vvsu_figure}

Функция \texttt{map} применяет лямбда-выражение \texttt{lambda obj: f'Protocol \{obj[0]\} - Criticality \{obj[1]\}'}, которое формирует строку для каждого кортежа. Результат преобразуется в список строк, соответствующих требуемому шаблону.

\subsection{Задание 9}

Необходимо отфильтровать список длительностей смен, оставив только те, что находятся в диапазоне от 8 до 12 часов включительно.

\begin{vvsu_figure}{Листинг программы для задания 9}{fig:code_task_9}
  \begin{minipage}{.75\textwidth}
    \lstinputlisting[language=Python,basicstyle=\fontsize{10}{10}\linespread{1}\selectfont\ttfamily]{code/task9.py}
  \end{minipage}
\end{vvsu_figure}

Функция \texttt{filter} с условием \texttt{8 <= x <= 12} отбирает подходящие значения. Например, из исходного списка \texttt{[6, 12, 8, 24, 10, 4]} будут выбраны \texttt{12, 8, 10}, так как они удовлетворяют условию. Результат — новый список, содержащий только допустимые длительности смен.

\subsection{Задание 10}

Требуется найти сотрудника с наивысшей оценкой и вывести его имя и балл.

\begin{vvsu_figure}{Листинг программы для задания 10}{fig:code_task_10}
  \begin{minipage}{.75\textwidth}
    \lstinputlisting[language=Python,basicstyle=\fontsize{10}{10}\linespread{1}\selectfont\ttfamily]{code/task10.py}
  \end{minipage}
\end{vvsu_figure}

Функция \texttt{max} принимает список словарей \texttt{evaluations} и ключ \texttt{key=lambda x: x['score']}, по которому происходит сравнение. Возвращается словарь сотрудника с максимальным баллом. Затем с помощью f-строки формируется и выводится сообщение с именем и оценкой. В данном случае выводится: «Сотрудник Dr. Weiss имеет высший балл равный 92».

\end{document}